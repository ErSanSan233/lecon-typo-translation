\chapter{参考资料:请认真参考}

\label{chap6}

参考资料是论文稿件、提交到期刊及现在提交到\emph{网络}平台的稿件中恒久远的错误来源。因此这里有一些用于避免大多数错误的原则。

首先,以下是期刊\emph{TSI——《计算机科学与技术》}(旧版)作者手册的一些摘录。

\begin{mdframed}
    参考资料有两个目的。

    \begin{enumerate}
        \item 致敬,诚实展现前人的作品,证明我们取得的进展;
        \item 告诉读者获得更多信息的方法。
    \end{enumerate}

    所有的参考文献都应是读者可以探索的。因此,我们在尽可能的情况下引用期刊文章而不是大学或公司的内部报告;如果会议已经产出集结成册的出版物,则应当引用该出版物,而不是发放给参会者的临时文件。

    要绝对避免形如``迪蓬,手稿''或``迪蓬,私人访谈''的参考资料。若有必要,这种类型的参考可以提前替换为脚注或致谢中的一个段落。
\end{mdframed}

优质的行文代表着需要给出完整和精确的参考文献,并由此在发表文章前就给出论据
    \footnote{我们都曾见过错误的参考资料通过复制粘贴从一篇文章传播到另一篇文章,我们也都不会忘记那些引用得很规范但我们没读过的参考资料。}
。对于今天的\emph{URL}来说,这一建议仍然适用:在将其放到文章中时,要去查验一下它是否还有效。

关于参考资料,有两个方面的内容需要说明:设置参考(如何调用和列出参考资料),以及参考资料的措辞。此处更注重后一方面。

\section{设置参考}

应用或给出参考资料的方式取决于期刊、图书、记录等的具体要求,无论针对纸质文件还是电子文件。需要遵守出版商给出的``作者手册''(如\emph{TSI}的手册为http://tsi.e-revues.com/)。总体来说,有两种方法。

\begin{enumerate}
    \item 第一种方法主要用于人文学科。这种方法要求将参考文献放在注释中(如置于页脚、侧栏,或在作品末尾重新分组列出)。这种方法可以将评论和参考的作用结合,当注释内容在页脚或侧栏时,我们可以立刻去参考它们,而不用去翻阅文章的其余部分(或直接点击跳转)。但当我们想要多次引用同一个资料却不想每次都重复时,我们就会使用一套相当抽象的话术,包含\emph{op. cit.}(同前)、\emph{id.}(同上)、\emph{ibid.}(同上上,若使用\emph{idem}则是\emph{ibidem})等。这里有一个典型的注释
        \footnote{[摘自Anthony \textsc{Grafton}的\emph{Les origines tragiques de l'érudition – une histoire de la note en bas de page},Le Seuil出版社, 巴黎, 1998, 注释8, 第194页] H.E. Davis, BA, \emph{An Examination of the Fifteenth and Sixteenth Chapters of Mr Gibbon's History of the Decline and Fall of the Roman Empire}, 伦敦, 1778年, 第\textsc{ii}页(由Gibbon引用,其画线于\emph{Miscellaneous Works}, 版本为John, lord Sheffield, 伦敦, 1814年, IV, 第523页)。Gibbon在其\emph{ Mémoires}中写道,Davis``声称攻击的不是信仰,而是历史学家的诚信''(\emph{op. cit.}, 第160页)。}
    。
    \item 在计算机科学中,往往使用第二种方法,即在正文中指出参考,并且在文章或作品的末尾重新组织这些参考资料(由第一作者字母序、日期顺序、引用顺序等顺序排列)。指出参考的方式可以是置于括号或引号中、使用上标等,表达方式可以是使用序号、缩写、全名……有时会精确到具体页。这种方式使得我们可以毫无困难地在列表中找到它。举例如下:
    
    \begin{quote}
        (PEN82)[Pendibidu][Pen82a, 第124页],mm$^{\rm pen82}$或像这样[4%TODO
        ]等。这些属于具体要求的范畴,我们不再探讨。
    \end{quote}
\end{enumerate}

\section{参考资料的标签}

无论是位于页脚还是全文末尾,参考资料都应遵循一定的用法,没有统一的范式
    \footnote{特别需要注意,在论文中编写参考文献跟在图书馆编写索引卡(或说明书等)是不同的,后者有范式(如Afnor Z),并且这些范式对于出版来说太过贫乏。}
,只有适合某家期刊或某家出版社的``做法''。我们需要遵循出版社的要求,但总体来说,有一些原则是通用的。

\begin{mdframed}
    基本原则如下:
    \begin{itemize}
        \item 作者名使用小型大写字母;
        \item 作品名使用意大利体,根据具体的情况,作品可能是书、论文、作品集(如研究报告集)、会议记录;
        \item 对于文章标题或数中的章节,置于引号之间;
        \item 用罗马体标识其他有用的信息(出版商、出版地、日期、卷、序号、页码等,不要忘记,现在还可能有\emph{网址})。
    \end{itemize}
\end{mdframed}

参考资料可以看作由一串结构化的元素串联而成,其中的每个元素都有特定的格式。事实上,一些排版工具就是这样工作的
    \footnote{比如需要搭配\LaTeX 使用的Bib\TeX(http://www.loria.fr/services/ctan/)。还有一些排版工具可供使用,比如Framemaker,甚至是一些XML产品。思路是只选取对当前正在编写的文章有用的参考资料,将它们编译成一个参考资料数据库。遵循这样的思路,后文\ref{sec6.2.1}%TODO
    小节中提到的图书可以这样定义:\\
    {\ttfamily 
    @BOOK\{nebut-unix\}\\
    author= "Jean-Louis Nebut"\\
    title = "UNIX pour l'utilisateur. Commandes et langages de commande"\\
    publisher="éditions Technip" place="Paris" date="1990" pagenumber="305".\\
    }
    这是根据文档所需的具体要求来组织的。}
。这样一来,产生排版错误的可能性就很小了。

以下是引用一些基本文档类型的示例。

\subsection{图书}

我们认为图书标题可以看作作品,因此使用意大利体。

\begin{quote}
    J.-L. \textsc{Nebut}, \emph{UNIX pour l'utilisateur. Commandes et langages de commande}, éditions Technip, Paris 1990, 305 p.
\end{quote}

\subsection{书中各章}

图书本身是作品,因此只有书名需要使用意大利体。

\begin{quote}
    G. \textsc{Lorette} et J.-P. \textsc{Crettez}, « Reconnaissance de l'écriture manuscrite », \emph{Traité d'informatique}, Techniques de l'Ingénieur, 1998, ch. H 1358, 18 p.
\end{quote}

\subsection{论文中的文章}

作品是论文本身,所以仅有论文使用意大利体。

\begin{quote}
    M. \textsc{Prevosto}, M. \textsc{Olagon}, A. \textsc{Benveniste}, M. \textsc{Basseville} \& G. \textsc{Le Vey}, « State space formulation, a solution to modal parameter estimation », \emph{Journal of Sound and Vibration}, vol. 148, n$^{\rm o}$ 2, 1990, p. 123-125.
\end{quote}

\subsection{会议报告}

组成作品的应当是会议记录(因此会议记录标题为意大利体)。

\begin{quote}
    Bertrand \textsc{Coüasnon} and Ivan \textsc{Leplumey}, ``A Generic Recognition System for Making Archives Documents Accessible to Public'', \emph{Proceedings of the Seventh International Conference on Document Analysis and Recognition (ICDAR'2003)}, (Apostolos Antonacopoulos ed.), IEEE Society, August 2003, p. 228-229; <http://csdl.computer.org/dl/proceedings/icdar/2003/1960/01/\linebreak1960010228.pdf>.
\end{quote}

\subsection{内部报告}

可以将系列名设置为意大利体。

\begin{quote}
    B. \textsc{Michel}, « Gothic Memory Management : a Multiprocessor Shared Single Level Store », \emph{Rapport de recherche Inria}, n$^{\rm o}$ 1202, mars 1990.
\end{quote}

\subsection{论文}

论文本身是作品,因此只有论文本身使用意大利体。

\begin{quote}
    T. \textsc{Daniel}, \emph{Traitement numérique d'images appliqué à l'analyse texturale de roches déformées}, Thèse de l'université de Rennes 1, 17 juin 1989.
\end{quote}

\section{几个注意事项}

\begin{itemize}
    \item 姓在前还是名在前,以及其组成形式(如使用括号、缩写等)取决于出版物的风格。
    \item 如果作者由多人,需要全部列出
        \footnote{复习一下:法文要写``Jean \textsc{Aymar}, Pierre \textsc{Quiroule} et Jean \textsc{Transen}''(只用一个逗号),英文要写``Jean \textsc{Aymar}, Pierre \textsc{Quiroule}, and Jean \textsc{Transen}''。}
    ,避免使用缩写\emph{et al.}。一方面,这种缩写不是法文的用法,另一方面,对于其他作者来说,这种缩写不是很友好
        \footnote{信息技术学科中,极难见到像粒子物理学那样有极多作者(甚至达到117位作者)的文献。}
    。
    \item 不要照搬图书封面上的版式(比如全部大写加粗),通常应当使用主标题页上的版式。添加必要的副标题同样重要。
    \item 可以在参考资料中使用约定俗成的缩写,无论是期刊名(如\emph{Journal of the Association of Computing Machinery}缩写为\emph{J. ACM}、\emph{Lecture Notes in Computer Science}缩写为\emph{LNCS})还是对期刊结构的引用(如chap.、vol.、p.、etc.)。结构的引用有时可以省略,因为可以根据上下文猜测(如1998, 6, 2, 12-23代表1998年第6卷第2期的12至23页)。
    \item ``页''的复数缩写是``p.''(英文为``pp.''),我们使用短横杠来分隔页码,如``p. 12-34''(英文中横杠略长,如``pp. 12--34'')。
    \item 最好遵循我们要引用出版物的所使用的语言的正字法,但法文出版物不会在参考资料中将外来术语设置为意大利体(除了作品名)。
    \item 如果要引用电子文章,不仅需要给出其\emph{URL},还要给出作者名、页面标题等,如参见第\ref{page47}页的参考资料\ref{ref9}。%TODO
\end{itemize}

\section{法文标题和外国出版物}

法国作者的参考资料往往因过于强调``法文不适合科技文本''而出错。在外国期刊上引用法文文章没什么丢脸的。以下是一些法文图书或文章被翻译成英文引用(或相反)的例子。

\subsubsection*{在英文期刊中}

\begin{description}
    \item[{[Montagné70]}] \textsc{Montagné} J., \emph{Théorie des graphes}, Dunœ ud, Paris, 1970. \{Translated into English as \emph{Graph Theory}, Adidas-Voley, 1974.\}
\end{description}

不要相信美国编辑不使用带变音符号的字母这种说法(不论是大写字母还是小写字母):如果你不加变音符号,他们不会为你凭空加上;但如果你加上了变音符号,他们会尊重你的做法。此外,请相信他们的编辑:如果你的引用写得太长,他们会为你改短;但如果你不明确要求,他们不会为你凭空加上法文的标题。

\subsubsection*{在法文期刊中}

请毫不犹豫地指出法文中该有的部件来帮助读者阅读。

\begin{description}
    \item[{[Montagné70]}] \textsc{Montagné} J., \emph{Théorie des graphes}, Dunœ ud, Paris, 1970.\\
    没有任何理由来将其翻译成英文。
    \item[{[Adore90]}] \textsc{Adore}, \emph{PreCursiv Reference Manual}, Adidas-Voley, Writing, 1990. \{Traduit en français : \emph{Manuel de référence du langage PreCursiv}, Dunœud éd., Paris, 1991.\}\\
    我们充分尊重了原作品,且使用法语的读者可以很方便地看到它的法文翻译。
    \item[{[Transen92]}] \textsc{Transen} J., « $\alpha$-laval results on the 887th leg of the centipede », \emph{Intern. Journal of Comput. Sc.}, 24, 8, 12–34, 1992. {voir aussi : « Résultats du test $\alpha$-laval sur la 887$^{\rm e}$ patte du mille-pattes », \emph{Publication interne de l'uPs}, n$^{\rm o}$ 98, juin 1991.}\\
    实际上,法国人发表的很多英文文章事先都有法文版本。所以,在引用它们时,也带上法文版本来为读者带来更好的阅读体验吧!
\end{description}