\chapter{参考资料:请认真参考}

\label{chap6}

参考资料是论文稿件、提交到期刊及现在提交到\emph{网络}平台的稿件中恒久远的错误来源。因此这里有一些用于避免大多数错误的原则。

首先,以下是期刊\emph{TSI——《计算机科学与技术》}(旧版)作者手册的一些摘录。

\begin{mdframed}
    参考资料有两个目的。

    \begin{enumerate}
        \item 致敬,诚实展现前人的作品,证明我们取得的进展;
        \item 告诉读者获得更多信息的方法。
    \end{enumerate}

    所有的参考文献都应是读者可以探索的。因此,我们在尽可能的情况下引用期刊文章而不是大学或公司的内部报告;如果会议已经产出集结成册的出版物,则应当引用该出版物,而不是发放给参会者的临时文件。

    要绝对避免形如“迪蓬,手稿”或“迪蓬,私人访谈”的参考资料。若有必要,这种类型的参考可以提前替换为脚注或致谢中的一个段落。
\end{mdframed}

优质的行文代表着需要给出完整和精确的参考文献,并由此在发表文章前就给出论据
    \footnote{我们都曾见过错误的参考资料通过复制粘贴从一篇文章传播到另一篇文章,我们也都不会忘记那些引用得很规范但我们没读过的参考资料。}
。对于今天的\emph{URL}来说,这一建议仍然适用:在将其放到文章中时,要去查验一下它是否还有效。

关于参考资料,有两个方面的内容需要说明:设置参考(如何调用和列出参考资料),以及参考资料的措辞。此处更注重后一方面。

\section{设置参考}

