\chapter*{注意}

在1990年,我写了一本名为《排版小课堂》的小书,其中收录了我在《Irisa周报》上以专栏形式发表的文章,我在其中``回顾''了一些排版的基本规则,并打算以此为基础编写一本科技文章写作手册。但我很快意识到这项工作的困难,甚至可以说是徒劳的,于是我就搁置了这个项目。

然而,《小课堂》被人以不太合法的方式(通常是盗版)上传到了网络上,并且格式不太便于访问(.dvi文件)。因此,有人要求我重新发布它们。在长时间的犹豫之后,我决定利用这个机会对它们进行重写、校对和扩充……在它们准备好之前,先放出这个版本——它在1990年原版的基础上做了轻微修改
    \footnote{主要的不同之处包括:版式、目录结构、增加了一些\emph{URL}、更新了参考文献……以及纠正了过去和未来的读者向我指出的一些录入错误(感谢读者们)。然而,我还没有增加任何补充内容,没有修改某些规则的表述,也没有重新审视我的某些原则选择来更贴近让-皮埃尔·拉克鲁(Jean-Pierre \textsc{Lacroux})的《正字法》(参见参考资料[\ref{ref5}],第~\pageref{ref5}~页)。%TODO
    }
。

我不能对那些被复制到其他网站上的版本的状态负责。只有本网站上的版本才能被视为最新版本:\link{http://jacques-andre.fr/faqtypo/lessons.pdf}
    \footnote{\emph{URL}、对参考资料的引用以及\textcolor{brown}{彩色}的脚注标记都是可点击的。}
。\label{aver}

\begin{flushright}
© 雅克·安德烈\\
© Jacques André\\
2003年10月版\\
此文档持续更新
    \footnote{以下人员提出了修正建议,在此一并感谢:\\
    Alem Alquier、Marwan Auger、Alain Bacquey、Guillaume Becq、Patrick Bideault、Stéphane Bortzmeyer、Guillaume Cabanac、Yves Cinotti、Antoine Colin、Pierre Dauchy、Armelle Domenach、Matthieu Dubuget、Laurent Douchin、Gilles Esposito-Farèse、Daniel Flipo、Alain Fossé、René Fritz、Fabien Galand、Xavier Gnata、Jean-Philippe Guérard、Henri Jabot、Michel Joubert、Charles Levert、Claude Lenormand、Vincent Lerouvillois、Jean-Baptiste Luciani、Nadine Martin、Jonathan Maurice、Olivier Miakinen、Geneviève Naud、Marc-André Oberholzer、Serge Paccalin、Patrick Percot、Olivier Pérès、Normand Perron、Éric Picheral、Jean-Baptiste Rouquier、Filippo Rusconi、Lionel de Sá、Arnaud Schmittbuhl、Gerhardt Stenger、Michel Taton、Patrice Tréton
    }
,最后一次更新是在\origindate。
\end{flushright}

\vfill
\begin{center}
    \emph{发现法文原书的任何错误请反馈至:
    }

    Jacques.AndreNN@gmail.com\qquad 其中NN=35
\end{center}

