\chapter{来写点高质量法文如何?}

研究人员越来越多地被要求使用英文来写作文章、报告等,去适应美国英语的规则(见参考资料%TODO
)。

但是,仍然有大量的情况允许我们或需要我们去使用法文,或者说,用正确的法文写作。如果说一些``规则''存在,那它们的目的就不是给你添堵,而是让文字更易懂、更易读。``写法文''包含

\begin{itemize}
    \item 遵循法文的拼写规则和语法,
    \item 以及使用正确的法文排印符号、适宜的缩写,等等。
\end{itemize}

其中,第一点不是本\emph{排印}小课堂的内容,但我想着重指出,现在可以找到五花八门的拼写检查工具,如MS Word集成的拼写检查功能、Ispell(法化版)等。

\begin{enumerate}
    \item 最大化使用这些工具,因为错别字总是让我们防不胜防。
    \item 充分了解它们的局限性:
    \begin{itemize}
        \item 它们通常会按照自认为正确的形式(字典上收录的形式)来纠正拼写,很少会考虑到性数配合;文章即使在MS Word上不被任何下线标红,也可能有很多配合上的问题。
        \item 最大化使用自定义词库来避免文中使用的科技术语不被识别或``被改错
            \footnote{我曾在一个科技小论文上见过一台``快到基于12年一次的循环的计算机'',MS Word无法识别纳秒(ns),将其改为了年(ans)!}
        ''的情况。
    \end{itemize}
\end{enumerate}

\section{法文字母}

今天的法文字母表包含42个(而不是26个)字母:

\begin{center}
    \Large a b c d e f g h i j k l m n o p q r s t u v w x y z \\
    à â é è ê ë î ï ô ù û ü ÿ ç æ œ
\end{center}

法文中,带有重音符的\emph{u}只在\emph{où}这一个单词中出现;带有分音符的\emph{u}很少出现
    \footnote{在aiguë、ambiguïté、contiguë等词中,分音符不出现在\emph{u}上,而是出现在其后紧随的元音上。}
,只在古法文或外来词(如capharnaüm、Bienvenüe等)中出现。带有分音符的\emph{y}(\emph{ÿ})出现在专有名词(L'Haÿ-les-Roses)或专有名词的派生词(aÿ,一种香槟)中。\emph{Œ}和\emph{œ}不是简单出于美学考虑的合字,我们不能统一将\emph{oe}替换成\emph{œ}(请想想``nœud coercitif''[强制节点]这个词组)。字母表中带有双字母\emph{æ}(罗马体写作æ),这是因为它在一些外来词(ægagropile、philæ等)或地名中出现。

请注意,法文中没有\emph{ñ}和\emph{ö},即使一些法文字典开始收录了cañon、angström、maelström等词。

每个字母都可以以3种形式使用,比如``aA\textsc{a}'',这代表:

\begin{enumerate}
    \item 小写(minuscule,也称作bas de casse)形式,如abéçô。
    \item 大写(majuscule,也称作capitale)形式,如ABÉÇÔ。
    \item 小型大写形式。简单来说,是看起来像大写字母的小写字母——\textsc{abéçô}。它用于表示首字母缩写、参考资料中的作者姓,以及一些结构上的元素(戏剧的对话、法律条文等)。
\end{enumerate}

变音符号在法国文化中占据一席之地,使用它们是非常重要的。哎呀,我们不是总能在打字机或计算机的键盘上``直接''找到这些字符(例如,我们通过\^{}\emph{o}来打出\emph{ô},至于\emph{É}嘛……)。字库中收纳了所有这些法文字符,没有任何理由不去使用它们(除了蠢和懒)。

\paragraph*{练习} 使用你喜欢的文字处理软件输入下面的全字母句
    \footnote{这句话由吉勒·埃斯波西托-法雷塞(Gilles Esposito-Farèse)设计。全字母句是指包含给定语言中的所有字母的句子,且长度越短越好。有一本C语言手册中用英文给出了这样一个示例:\texttt{char *MyString = "The quick brown fox jumps over the lazy dog";}。译者显然什么都不懂,直接将其翻译成了\texttt{char *ZiFuChuan = "那只敏捷的棕毛狐狸跃过那只懒狗";}。想要在法文中保留全字母句,至少也应当翻译成\texttt{"Portez ce vieux whisky au juge blond qui fume"(把那杯老威士忌带给抽烟的金发法官)}……
    }
(先使用小写字母,再使用小型大写字母)。

\begin{center}
    Dès Noël où un zéphyr haï me vêt de glaçons würmiens\\
    je dîne d'exquis rôtis de bœuf au kir à l'aÿ d'âge mûr\\
    \& cætera !

    \textsc{Dès Noël où un zéphyr haï me vêt de glaçons würmiens\\ je dîne d'exquis rôtis de bœuf au kir à l'aÿ d'âge mûr\\ \& cætera !}
\end{center}

如果你做不到,必须去更换系统……

\section{一些排印符号}

文本不只包含字母、数字和标点(见第\ref{chap5}章),还包含一整套跟随语言而特定变化符号。以下是一些法文符号。

\begin{description}
    \item[法文引号] 在法文中,引号是一对双楔形符号«...»,而不是英文的\emph{(双)蝌蚪形符号}``...''或`...'(见图\ref{fig1}第5行)。
    \item[连接号] 我们要区分使用3种连接号:\\
    - 用于连词或行末\\
    -- 用于标示减号\\
    --- 用于标示插入语或列表元素,却有被中间的``--''取代的趋势。
\end{description}

相反地,以下符号需要避免:

\begin{description}
    \item[$\bullet$] 美国化符号,太粗大;需要使用--或---替代(见图\ref{fig1}第14行);
    \item[/] 这个斜杠被(过于)经常地作为``或''的含义使用;在文本中更倾向于使用连词``或''(ou),它只多占了一个字符。\\
    在什么情况下,我们都不用``或/和'' ``和/或''这样的表达,它们不比一个``或''字包含任何更多信息:在法文中,\emph{或}不是不相容的(见图\ref{fig1}第9行)。
\end{description}

\section{缩写}

一些缩写已经约定俗成,需要坚持使用下去,其中大多数不需要大写:用art.代表article,用vol.代表volume,等等。它们也不需要加复数后缀。但是,一些缩写的单数和复数形式不同(例如,M.代表monsieur,其复数messieurs的缩写为MM.),需要在行文中避免这样的缩写。以下是一些需要了解的常用缩写(可参阅任何排印语言良好的长列表,参阅参考资料%TODO
)。

\begin{center}
    \begin{tabular}{|l|l|}
        \hline
        artivle(文章/条款)& art.\\
        bulletin(通报)& bull.\\
        tome(卷)& t.\\
        page, pages(页,单复数)& p. \\
        numéro(序号,单数)& n$^{\rm o}$  N$^{\rm o}$ \footnotemark\\
        numéros(序号,复数)& n$^{\rm os}$ \\
        document(文档)& doc. \\
        édition(版)& éd.  \\
        sous la direction de(指导) & sld(英:\emph{ed.})\\
        et collaborateurs(等人)& et coll.(不用\emph{et al.})\\
        note de la rédaction(编者按)& n.d.l.r.\\
        \emph{confer}(= voir;参考)& cf.(用罗马体)\\
        c'est-à-dire(即)& c.-à-d.(不使用c.a.d或\emph{i.e.})\\
        Monsieur(先生)& M.  \\
        Madame(女士) & M$^{\rm me}$\\
        \hline
    \end{tabular}

    \footnotetext{上标使用字母而非度号}

\end{center}

\begin{center}
    \begin{tabular}{|l|l|l|}
        \hline
        形容词 & 缩写\footnotemark & 不要使用 \\
        \hline
        premier(第一,阳性单数)& 1$^{\rm er}$, 1er & 1$^{\rm ier}$, 1ier \\
        première(第一,阴性单数)& 1$^{\rm re}$, 1re & 1$^{\rm ière}$, 1ière, 1ere \\
        premières(第一,阴性复数)& 1$^{\rm res}$, 1res & 1$^{\rm ières}$, 1ières \\
        deuxième(第二)& 2$^{\rm e}$, 2e & 2$^{\rm ième}$, 2$^{\rm eme}$,  2ème, 2è \\
        \hline
    \end{tabular}
    
    \footnotetext{在极少数字母上标不可用的情况(如电子邮件)下,不使用上标的情况是可以容许的。}
\end{center}

\begin{center}
    \begin{tabular}{|l|l|}
        \hline
        副词 & 缩写\\
        \hline
        primo(首先) & 1$^{\rm o}$ \\
        secundo(其次) & 2$^{\rm o}$ \\
        tertio(其三) & 3$^{\rm o}$ \\
        quarto(其四) & 4$^{\rm o}$ \\
        \hline
    \end{tabular}
    
    \footnotetext{在极少数字母上标不可用的情况(如电子邮件)下,不使用上标的情况是可以容许的。}
\end{center}

\section{单位}

研究人员不知道怎么写测量单位——请你相信这个结论,我不想去费时费力地解释它。在这里,我只给出几个例子:2安培应当写作deux ampères,不应使用Ampère或Ampères,2~A的写法也是正确的;2,34~kg是正确的,2.34~Kilos是错误的,2,34~Kgrs更是错得离谱;17~F是正确的,17~Frs是错误的。请参阅参考资料%TODO
。

\section{断字}

行末单词的断字(称作coupure、césure或division[最后一种叫法更好些])的工作通常是由文字处理系统处理的,它们在多年前就已经在这方面取得了不少进展。但不论如何,它们偶尔还是会出些问题(见图\ref{fig1}第11行),甚至引入差错。因此应该由作者去确保断字正确,不要打断不该打断的单词。

然而,目前的文字处理系统很少能够处理单词间的断字问题,尤其是数字和紧随其后的单位间、人名首字母和姓之间不应换行(见\ref{fig1}第3行)。这就需要用到一个不可分空格(une espace
    \footnote{这里的espace是阴性名词。在铅字排印时期,这个词的阴性指能在一行文字间产生用于单词或符号的``空白''的一个字符。现在,即使字符已``去金属化'',法文仍然延续了这个用法,将能够产生空白的字符作为阴性名词使用。}
insécable;见\ref{sec5.1.3}小节)。

注意:在引用一段外文时,应当使用改文种的用法来确定断字规则。以下是一个例子:

\begin{quote}
    ……单词潜意识识别(英文称作\emph{word sub-\\
    liminal recognition})……
\end{quote}

\section{美国化}

受教于美国出版物,人们趋于相信其中的各种用法组成了一套自己的规则,并东施效颦,引入各种美国化的内容。然而,其中一些用法本有对应的法文。

\paragraph*{拉丁文短语} 以英语为母语的人笔下的那些短语,即使来源于拉丁文,也不符合法文的用法。

\begin{center}
    \begin{tabular}{|l|l|}
        \hline
        避免使用 & 使用 \\
        \hline
        \emph{e.g.} & p. ex. \\
        \emph{et alii, et al.} & et co-auteurs, et coll., etc.\\
        \emph{id est, i.e.} & c'est-à-dire, c.-à-d.\\
        versus, vs & contre, « - » \\
        \hline
    \end{tabular}
\end{center}

\paragraph*{缩写} 不同国家的缩写不同。例如:

\begin{itemize}
    \item 页(复数):法文使用p.,英文使用pp.;
    \item 先生:法文使用M.,英文使用Mr。
\end{itemize}

\paragraph*{正字法} 有很多区别,举例如下:

\begin{itemize}
    \item 标点符号(英文中,分号、冒号等符号前不加空格);
    \item 引号(英文为``...'',法文为« ... »);
    \item 法文中,标题的各实词首字母不需要大写,等等。
\end{itemize}

http://www.panamo.com/RESS/anglais.html 列出了很多其他区别。