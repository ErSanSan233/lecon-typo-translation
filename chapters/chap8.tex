\chapter{拓展阅读}

以下是一些参考资料(更新于2004年),它们可以让你更进一步。几乎所有以下作品都可以在Irisa的图书馆中找到,但其中的一些应当一直放在你的手边或鼠标边。

\section{排印要求}

虽然法兰西学院还留着它的\emph{《字典》},但在法国,还没有官方机构在追寻排版规则的改变。然而,对于法文,可以参考这些作品:

\renewcommand{\labelenumi}{[\theenumi]}

\begin{enumerate}
    \item \emph{Code typographique – Choix de règles à l'usage des auteurs et des professionnels du livre}, Fédération nationale du personnel d'encadrement des industries polygraphiques et de la com- munication, (64, rue Taitbout, 75009 Paris), 13$^{\rm e}$ édition, 1981.\\
    (该书一度被誉为校对圣经,但它已经绝版,且有些过时了 [20世纪90年代出了新版,新版简直是场灾难] 。)
    \item \emph{Lexique des règles typographiques en usage à l'Imprimerie nationale}, Imprimerie nationale, 2004 (4$^{\rm e}$ édition).\\
    (目前,该书时最容易获取且我最推荐的。) \label{ref2}
    \item \emph{Guide du typographe -- Règles et grammaires typographiques à l'usage des auteurs, éditeurs, compositeurs et correcteurs de la langue française}, édité par le Groupe de Lausanne de l'Association suisse des compositeurs à la machine, 6$^{\rm e}$ édition, Lausanne, 2000.\\
    (我认为该书查阅起来比国家印刷馆的[\ref{ref2}]更完整。)
    \item Aurel \textsc{Ramat}, \emph{Le Ramat de la typographie}, Aurel Ramat éditeur, Montréal (Québec), 2002 ; I.S.B.N. : 2-922366-01-4.\\
    (比上一个\emph{Guide}还好,但也更难获得。)
\end{enumerate}

这些作品的细节往往不一致甚至冲突。一位作者尝试基于全球性而不是零散的规则对它们进行分析和整合。在他过世后,其未完成的作品被放到了网上:

\begin{enumerate}[resume]
    \item Jean-Pierre \textsc{Lacroux}, Orthotypographie, 2007, <http://www.orthotypographie.fr>.
\end{enumerate}

我们也可以查询一些不太完整的作品,比如以下这些:

\begin{enumerate}[resume]
    \item \emph{Le style du Monde}, SA Le Monde, 2004, I.S.S.N : 0153-419X.
    \item Éric \textsc{Martini}, \emph{Petit guide de typographie}, Glyphe, Paris, 2003 ; <http://www.glyphe.com/typo.pdf>.
\end{enumerate}

再比如一些通用的排印学作品,比如[\ref{ref21}]、[\ref{ref23}],它们通常给出了基本的规则。此外,还可以查询网络上的以下讨论:

\begin{itemize}
    \item Liste typo : <https://www.irisa.fr/wws/info/typographie> ;
    \item Forum langue française : <http://langue-fr.net/> ;
    \item Blog des correcteurs du \emph{Monde} (« Langue sauce piquante ») :
    <http://correcteurs.blog.lemonde.fr/>。
\end{itemize}

\section{法文以外的规则}

\begin{enumerate}[resume]
    \item \emph{The Chicago Manual of Style}, The University of Chicago Press, 16$^{\rm e}$ édition, 2003.\\
    (针对英文的参考资料。它有一个问答网站 [问题类似于\emph{URL的换行是否应当使用硬空格?}] ,见<http://www.chicagomanualofstyle.org/home.html>。)
    \item Dominique \textsc{Lacroix}, \emph{Mémento de typographie anglaise à l'usage de rédacteurs francophones}, <http://www.infotheque.info/cache/9491/www.panamo.com/RESS/anglais.html>.\\
    (很好的页面,带有很好的链接。)
    \item Office des publications de l'Union européenne, Code de rédaction interinstitutionnel : <http://publications.eu.int/code/fr/fr-000300.htm>.
\end{enumerate}

\section{公制系统}

现在叫国际单位制(Système international d'unités,SI)。

\begin{enumerate}[resume]
    \item Organisation intergouvernementale de la convention du mètre, \emph{Le Système international d'unités (SI)}, édité par le BIPM, pavillon de Breteuil, Sèvres, 8e édition, 2003. Voir aussi <http://www1.bipm.org/fr/si/si\_brochure/>. \label{ref11}
    \item Michel Dubesset, \emph{Le manuel du Système international d'unités}, Éditions Technip, 2000.\\
    (非常详尽,带有量纲式。)
    \item \emph{International System of Units (SI)}, The Physics Laboratory of NIST, <http://physics.nist.gov/cuu/Units/index.html>.
    (适合那些更喜欢用英文而非法文阅读国际标准[\ref{ref11}]等的人。)
\end{enumerate}

\section{法文}

除了各种字典外,可以按需查询以下作品。

\begin{enumerate}[resume]
    \item Jean-Claude \textsc{Colin}, \emph{Dictionnaire des difficultés de la langue française}, Robert, coll. « Les usuels du Robert », 1988 (et son annexe : \emph{Dictionnaire typographique} de Jean-Yves \textsc{Douron}).
    \item Jacques \textsc{Drillon}, \emph{Traité de la ponctuation française}, éditions Gallimard, Paris, 1991.
    \item Jean \textsc{Girodet}, \emph{Pièges et difficultés de la langue française}, Dictionnaire Bordas, Paris, 1986 (4$^{\rm e}$ édition).
    \item \textsc{Grevisse}, \emph{Le bon usage}, éd. Duculot, Belgique ; (révisions quasiannuelles, par André Goosse).
    \item V. \textsc{Thomas}, \emph{Dictionnaire des difficultés de la langue française}, Larousse éd., 1956.
\end{enumerate}

也可以参见``langue française''论坛网站:<http://www.langue-fr.net/>。

\section{版式、字体等}

这里有一些关于``经典''排印(字体、版式等)的作品:

\begin{itemize}[resume]
    \item Pierre \textsc{Duplan}, Roger \textsc{Jauneau} et Jean-Pierre \textsc{Duplan}, \emph{Maquette et mise en page}, Éditions du Cercle de la Librairie, Paris, 2004, I.S.B.N. : 2-7654-0876-9. \\
    (一部经典作品的再版,适应网络需求。)
    \item James \textsc{Felici}, \emph{Le Manuel Complet de Typographie}, PeachPit Press, Paris, 2003, 320 pages ; I.S.B.N. : 2-7440-8067-5.\\
    (很完整……但很美式,尤其是示例。)
    \item Damien \textsc{Gautier}, \emph{Typographie, guide pratique, Pyramyd}, seconde édition 2001, I.S.B.N. : 2-910565-16-5.\\
    (针对版式和字体方面的解释很好。)
    \item Yannis \textsc{Haralambous}, \emph{Fontes et codages}, O'Reilly, Paris, 2004, I.S.B.N. : 2-84177-273-X.\\
    (对于想要知道字体如何起作用的人来说,是一本基础且不可或缺的书。)
    \item Yves \textsc{Perrousseaux}, \emph{Manuel de typographie française élémentaire}, Atelier Perrousseaux éd., 1996, I.S.B.N.\\ 
    (入门必读书。)
    \item Robert \textsc{Bringhurst}, \emph{The Elements of Typographic Style}, Hartleys \& Marks, publishers, 2$^{\rm e}$ édition 1996, I.S.B.N. : 0-88179-132-6.\\
    (关于字符的书,十分适用于法文……)
    \item Jorge De Buen, Manual de diseño editorial, Santilano, Mexico, 2000, I.S.B.N. : 970-642-655-8.\\
    (很难找到,且是西班牙文书。此处引用该书是因为它实在是一本典型的好书,但没有法文版……)
\end{itemize}

也请参见<http://jacques-andre.fr/faqtypo/biblio.pdf>中列出的参考资料。

\vfill

\begin{center}
    \rule{.75\linewidth}{.5pt}

    \textbf{版本记录:}原法文文件由Lua\LaTeX 编译,字体:\emph{TeX Gyre Pagella}。
\end{center}
