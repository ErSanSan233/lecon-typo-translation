\chapter{\textmd{\underline{加下线}、\textbf{加粗}还是\emph{使用意大利体}?}}

受制于技术原因,铅字时代的排印不存在添加下线的操作,它是到了打字机时代才出现的发明。实际上,由于无法打出粗体、意大利体,打字员大量使用下线。这种做法形成了一种文档风格,在今天还在影响着研究人员的报告,因为人们总是倾向于去重复曾经和当下看到的事物。然而,我们不能将下线和粗体、意大利体等同起来,从而互换使用。

\section{原则}

在分别介绍它们的用法之前,先来简要介绍一下阅读文档的模式。大体上来说,可以将文档分为两大类:线性阅读文档(如小说、文章、研究报告等),以及不必从头读到尾,而是去查询的文档(年鉴、目录、参考手册等)。前一种文档的排印应当是冷淡一些(在一页文本中,视线不会立刻被某个词或某个标记抢走);第二种文档的排印则应当跌宕起伏(使得视线可以快速锁定到正在寻找的内容)。当然,这种分类方式很不明确,一些作品可以同时属于两类,带有教学性质的作品尤为如此。

在详细说明前,先给出总体框架:

\begin{mdframed}
    粗体只用于两种情况:章节等的标题,以及在参考手册或目录中的入口标记。

    \noindent 意大利体表示区别:外语、强调、引用、作品名(包括参考资料中的书名和报刊名)。

    \noindent 通常情况下,没有任何理由使用下线。
\end{mdframed}

罗马体、粗体、意大利体、下线的侧重点不同:

\begin{itemize}
    \item 罗马体(字符笔直,如同此处的形式)是通常的行文字体,可读性一般较强。
    \item 意大利体是一种倾斜的字体(但其字母的形态通常与对应的罗马体不同,如``\emph{fa}''不是将``fa''倾斜而来)。这种倾斜的形态使得我们在罗马体的句子中看到意大利体时能够察觉区别(因此具有潜在的含义)。相反,当我们看到一页文字(比如此页)时,我们无法立刻注意到是否有使用意大利体的内容。因此,我们使用意大利体来体现不需要具有吸引视线功能的差异。\\
    一般认为,意大利体的可读性不如罗马体,整段的意大利体文字读起来比整段的罗马体文字吃力。此外,我们只在个别词上使用意大利体。如果使用得太多,意大利体将失去其``突出显示''的作用。
    \item 字符的粗体依靠其笔画和厚重程度实现。罗马体和意大利体多少都可以呈现出粗体的形态。对于罗马体的加粗变体通常称为``加重''(gras,直译为``油腻'')。将字体加粗,可以使其更显眼,从而更吸引目光。如果在一整段罗马体文字中夹带一个加粗的词(比如本段文字),在阅读时,你会立刻被\textbf{加粗}的词吸引。因此,我们使用加粗来吸引视线,尤其是指示文章结构(章节标题等),或者标识目录的入口、咨询手册中的定义等。\\
    加粗字符越多,则文本的可读性越差。因此,我们只在广告中使用很粗的文字。
    \item \underline{加下线}也可以吸引视线,且比加粗的效果更显著。此外,由于下线可能切断下探的笔画,字符会尤其难读。因此,极少有使用下线的充分理由。建议永远不要使用下线。特别要注意,我们已经被\emph{网}上超链接的下线养成了随手使用它的坏习惯。
\end{itemize}

现在,以颜色突出显示的成本很低(无论是在屏幕上还是在纸上)。因此,它也有被滥用的趋势,正如几年前切换字体的成本变低,导致我们曾滥用字体一样。我想说:``请像谨慎使用粗体那样谨慎使用颜色。''``请避免使用蓝色文字,将蓝色留给网址。''

\section{章节等标题}

字符的选择(字体、字重、字号)取决于版式布局,通常需要由专业人士来完成(参见第~\pageref{sec8.5}~页列出的几本书)。尽管如此,至少要遵循以下用法:

\begin{itemize}
    \item 通常,标题需要加粗。
    \item 层级更高的标题通常字号更大(例如,章标题字号为18,节标题为14,小节标题为12,正文为11,等等)。
    \item 这种渐进的选择应当足够显眼,使得即使没有序号,读者也能区分文档的结构,同时也不能过于夸张以至于扰乱阅读节奏(例如,请不要将节标题字号设置为18的同时将正文字号设置为10)。
    \item 除非情况特殊,否则标题层级不要超过三层或四层。此时,我们可以将最低级标题设置为意大利体,将正文设置为罗马体。
    \item 为加粗字符添加下线完全没必要(见图~\ref{fig1} 的第17行),没必要,没必要。
    \item 跟加粗同样重要的是标题前后的空白大小,但这是另一个问题了……
\end{itemize}

\section{入口和定义}

在目录或参考手册这样的文档中,推荐使用粗体来``牵制''视线。例如:

\begin{quote}
    并非所有参考资料的排版都相同。事实上,参考资料的排版取决于其性质:

    \textbf{图书:}书名需要使用意大利体,其余部分,包括(且必须包括)作者名、出版商、出版地点、出版日期,使用罗马体。
    
    \textbf{文章:}作者名使用罗马体;文章标题使用罗马体且置于引号间;期刊标题使用意大利体;其余部分,如卷号、序号、页码等,使用罗马体

    \textbf{论文:}与图书相同,但要将出版商替换为大学名。

    \textbf{等等。}
\end{quote}

定义可以使用意大利体或粗体。使用意大利体显得更审慎些,会更好地融入正文。使用粗体则更容易被识别(尤其是末尾用于跳转到定义所在页码的索引),但更吸引眼球也意味着会破坏线性阅读的体验。例如:

\begin{quote}
    Le système est organisé autour de couches : la première couche, autour du matériel, constitue ce qu'on appelle le \emph{noyau} du système. C'est lui qui réalise les \emph{échanges avec les périphériques} et qui gère les tâches (y compris la mémoire).

    Le système est organisé autour de couches : la première couche, autour du matériel, constitue ce qu'on appelle le \textbf{noyau} du système. C'est lui qui réalise les \textbf{échanges avec les périphériques} et qui gère les tâches (y compris la mémoire).

    \begin{bil}
        系统是围绕层组织的:围绕硬件的第一层构成了所谓的系统\emph{核心}。它执行\emph{与设备的交换}并管理任务(包括内存)。

        系统是围绕层组织的:围绕硬件的第一层构成了所谓的系统\textbf{核心}。它执行\textbf{与设备的交换}并管理任务(包括内存)。
    \end{bil}
\end{quote}

\section{计算机术语}

Algol 60的报告开创了在程序中加粗关键字的习惯,然而,使用意大利体本应是更好的做法。然而,习惯已经养成了,现在说这些太晚了!然而,我还是建议不要在指令、说明等文本中过度使用加粗字符。请比较以下两个表述:

\begin{quote}
    Cette fonction est plus difficile à mettre en œuvre. Nous créons deux types d'attributs \textbf{contact\_haut} et \textbf{contact\_bas} pour réaliser cette fonction \textbf{poser\_sur}. Comme nos primitives sont limitées au cube, à la sphère et au cylindre, \textbf{poser\_sur} a en fait trois possibilités...

    Cette fonction est plus difficile à mettre en œuvre. Nous créons deux types d'attributs \emph{contact\_haut} et \emph{contact\_bas} pour réaliser cette fonction \emph{poser\_sur}. Comme nos primitives sont limitées au cube, à la sphère et au cylindre, \emph{poser\_sur} a en fait trois possibilités...

    \begin{bil}
        此函数更难实现。我们将创建两个属性\textbf{contact\_haut}和\textbf{contact\_bas}来实现此\textbf{poser\_\linebreak sur}函数。由于原函数限制在正方体、球体及圆柱上,\textbf{poser\_sur}实际由三种情况……

        此函数更难实现。我们将创建两个属性\emph{contact\_haut}和\emph{contact\_bas}来实现此\emph{poser\_sur}函数。由于原函数限制在正方体、球体及圆柱上,\emph{poser\_sur}实际上有三种情况……
    \end{bil}
\end{quote}

\section{术语}

\begin{itemize}
    \item 术语、我们想要突出显示的词、定义,总之,就是我们想要画重点(但不要真的画线上去)的内容,使用意大利体。\\
    \begin{quote}  
        Pour faire avancer la simulation, il faut que $\delta$ puisse déterminer que dans ces conditions \emph{aucun message ne sera émis} avant l'instant $\theta + 5$. Pour pouvoir libérer de la place en mémoire, on utilise la notion de \emph{temps virtuel global} : à un instant donné...
        \begin{bil}
            为了推进模拟,$\delta$必须能够确定在这些条件下,在时间$\theta + 5$之前\emph{不会发送任何消息}。为了释放内存空间,我们使用了\emph{全局虚拟时间}的概念:在给定时刻……
        \end{bil}
        
        Les électeurs sont donc invités à voter \emph{oui} lors du prochain scrutin.
        \begin{bil}
            因此,鼓励选民在下次投票中投\emph{赞成}票。
        \end{bil}
    \end{quote}
    \item 通常,用意大利体强调的内容也可以用引号强调,但使用引号时,意大利体应当改为罗马体。这是因为同时使用意大利体和引号相当于强调了两遍,是多余的。
\end{itemize}

\section{外来词}\label{sec4.6}

\begin{itemize}
    \item 外来词应当使用意大利体(不要添加多余的引号)。例如:
    \begin{quote}
        Le projet a joué un rôle primordial dans la seconde \emph{International Conference on Supercomputing} qui s'est tenue à Saint-Malo en juin.
        \begin{bil}
            该项目在6月在圣马洛举行的第二届\emph{国际超级计算会议}上发挥了关键作用。
        \end{bil}

        Le système Mentoniezh (du breton \emph{ment}, mesure, et \emph{oniezh}, science de, c'est-à-dire géométrie)...
        \begin{bil}
            Mentoniezh系统(源自不列颠语\emph{ment} [意为测量]、\emph{oniezh},[意为``……的科学''],即地理)……
        \end{bil}

        Pour notre circuit, l'autocadencement (\emph{self-timing}) se révèle bien adapté...
        \begin{bil}
            对于我们的电路,自动计时(\emph{self-timing})非常适合……
        \end{bil}

        Jean Transen collabore avec le \emph{VLSI Research Group} de l'université d'Oxford.
        \begin{bil}
            让·特航桑与牛津大学\emph{VLSI研究小组}合作。
        \end{bil}
    \end{quote}
    \item 外来的特定表达使用意大利体,如\emph{a capella}(纯人声演唱)、\emph{de facto}(事实上)、\emph{for ever}(永远)、\emph{honoris causa}(名誉上的)、\emph{ipso facto}(根据事实地)、\emph{manu militari}(武力地)、\emph{sine die}(不定期地)、\emph{up-to-date}(最新)等。
    \item 对应地,一些外来的表达在使用时保持罗马体。至于如何界定是罗马体和意大利体的界限,则有待讨论……例如:à priori
        \footnote{因此此处使用罗马体,甚至带了重音符。参见Lactoux的作品[\ref{ref5}, art. Latin]。\label{note19}}
    (先验地)、ad hoc(专为此时的)、andante(行板)、curriculum vitæ(简历)、ex æquo(并列)、fair play(公平竞争)、mea culpa(捶胸悔过)、sketch(幕间短剧)、statu quo(现状)、vice versa(相反地)、etc.(等等)。(表示``等等''的etc.当然也是这个列表的一部分。)列表可能有不同,需要去看具体的版式要求。
    \item 在参考书目中的一些拉丁文(往往是首字母缩略词)需要使用意大利体(这一点在人文学科中比在计算机学科中更常见):\emph{passim}(及其他位置)、\emph{op. cit.}(同前)、\emph{infra}(下文)、\emph{ibidem}(同书),等等。
\end{itemize}

\section{引用}

引用其他作者的话的惯例时区分他的内容和你的内容。通过排印学方法实现这一目的的方法有很多,比如使用引号、意大利体,或者通过字符的变化来使得页面的版式不同。当引用的内容比较短,或者需要使用特殊版式(比如带有标题、日期、收件人地址的信件)时,我们通常使用意大利体(而不带引号)。意大利体也会用于表示演讲稿、格言等。例如:

\begin{quote}
    Jean-Marie Pendibidu l'a bien dit : \emph{Le coût du matériel décroît rapidement tandis que le coût de développement du logiciel ne cesse de croître.} Donc, ...

    \begin{bil}
        让·马里·庞迪毕居说过:\emph{硬件成本正在迅速下降,而软件开发成本正在稳步增长。}因此,……
    \end{bil}
    
    Puisque \emph{tant va la cruche à l'eau qu'elle se casse}, je propose que nous nous arrêtions...

    \begin{bil}
        俗话说\emph{瓦罐不离井边碎},我建议停止……
    \end{bil}

\end{quote}

有时,如果我们需要引用同时带有意大利体和罗马体的内容(比如引用上述内容),则可以使用引号。

技术文档经常需要引用一个字母或一个词,此时意大利体可以应付一切(且没有引号那么繁重)。请比较以下三句话:

\begin{quote}
    La lettre a a une panse, mais pas c ni n. Paris a cinq lettres.

    La lettre \emph{a} a une panse, mais pas \emph{c} ni \emph{n}. \emph{Paris} a cinq lettres.

    La lettre « a » a une panse, mais pas « c » ni « n ». « Paris » a cinq lettres.

    \begin{bil}
        字母a带有字肚,但c和n没有。Paris有五个字母。
    \end{bil}
\end{quote}

\section{作品名}

\begin{itemize}
    \item 通常,文学作品、艺术作品、科学作品等作品的名称使用意大利体。同样地,船名、飞机名使用意大利体。排印规则中使用了10页左右的篇幅来说明标题中可能出现的冠词是否应当使用意大利体。基本原则是:如果冠词是作品名的一部分,则使用意大利体。举例如下:
    
    \begin{quote}
        Outre \emph{Les caractères} de La Bruyère, \emph{La Jument verte} de Marcel Aymé et les \emph{Fables} de La Fontaine sont mes livres de chevet.

        \begin{bil}
            除了拉布吕耶尔的《品格论》,马塞尔·埃梅的《绿色的母马》和拉·封丹的《寓言》)也是我的床边书。
        \end{bil}

        À la maison de la culture, j'ai entendu le duo de Manon, la marche de \emph{L'or du Rhin}, la valse de \emph{Faust} et \emph{le Concerto No 3 en si bémol de Saint-Saëns} (bien meilleur que celui en \emph{la majeur}).

        \begin{bil}
            在文化中心,我听过《曼侬》二重唱、《莱茵的黄金》进行曲、《浮士德》华尔兹,以及圣-桑的《b小调第三协奏曲》(比《大调协奏曲》要好得多)。
        \end{bil}

        On peut aussi y voir une copie du \emph{Guernica} de Picasso et une de \emph{L'Angélus} de Millet.

        \begin{bil}
            你还可以看到毕加索的《格尔尼卡》和米勒的《天使》。
        \end{bil}

        J'ai voyagé sur \emph{La Belle Poule} et le \emph{France} puis je suis revenu en Concorde.

        \begin{bil}
            我乘坐贝勒·普尔号和法兰西号旅行,然后回到协和式飞机上。
        \end{bil}
    \end{quote}
\end{itemize}

\section{书刊名}

\begin{itemize}
    \item 书刊都属于作品,因此使用意大利体:Ni \emph{Le Monde}, ni \emph{Ouest-France }ne parlent du \emph{Capital}...(《世界报》和《法兰西西部报》都没有提到《资本论》……)
    
    在列举参考书目时,这一点尤其重要。见第~\ref{chap6}~章。
\end{itemize}

\section{数学变量}

\begin{itemize}
    \item 数学变量名和函数名使用意大利体书写(除了常量、三角函数、自然对数的底、极限等,它们仍然保留罗马体)。举例如下:
    
    \begin{quote}
        变换$h(x, y) = {\rm sinc}\,x\,{\rm sinc}\,y$,其中$ {\rm sinc}\,x=\sin(\pi f_S x)/\pi f_S x$……
    \end{quote}
\end{itemize}

\section{字体名}

字体(fonte,也称police、caractère)也是一种作品,所以字体名使用意大利体(就像葡萄酒一样
    \footnote{很多计算机科学家都喜欢葡萄酒。以下是资深排印师、酿酒师让-德尼·龙迪内(Jean-Denis Rondinet;见``排印清单'',参见%TODO
    )关于酒庄名称的看法:``总的来说:
    \begin{enumerate}[label=\alph*)]
        \item 强调酒是拿来喝的时,酒庄名用小写:un verre de château-trompette(一杯特龙佩特酒庄的酒);j'ai acheté pour 100 € de château-trompette(我花100欧元买了特龙佩特酒庄的酒);
        \item 强调那种爬满常春藤的建筑时,酒店名用一个大写字母:le château Trompette est à 2 km de chez moi(特龙佩特酒庄距离我家2 km;注意,这可能代表村子、城区,或者品牌,请记得核实);vin du château Trompette(特龙佩特酒庄产的酒)、vin de Champagne(香槟地区产的酒) ,但香槟酒是du champagne、波尔多酒是du bordeaux;
        \item 强调股票,则多用大写字母和div%TODO 不知道div是啥
        :Château-Trompette entre au second marché(特龙佩特酒庄打入第二市场);j'ai acheté pour 1 000 € de Château-Trompette(我买入特龙佩特酒庄1000欧元);
        \item 任何花哨的商业名称(Domaine [特指勃艮第地区的酒庄]、Cuvée [特酿]……每天还有更多这种词出现)都是应当大写的专有名词: une gourde de château-trompette 1988 Domaine de la Reine-Blanche(一葫芦特龙佩特酒庄1988年勃艮第白皇后庄园的酒);un verre de trompette rosé Cuvée des Seigneurs récolte tardive(一杯特龙佩特玫瑰红晚收王侯特酿);
        \item 一些牌子不是葡萄酒牌子:un cubi de Postillon(一盒Postillon)、un carton de Kiravi(一箱Kiravi)、une bouteille de Veuve Cliquot(一瓶凯歌香槟);
        \item 复数:一般名词加复数(des pommards [几个勃艮第红葡萄酒]、des bourgognes aligotés [几个勃艮第白葡萄酒]),但复合名词和品牌不加复数(des sainte-croix-du-mont [几个圣十字山酒]、des Gévéor [几个热韦尔酒]……)。
    \end{enumerate}}
),且首字母不大写。

当我们指代一个字母或一个字符时,与其将其放到引号间,不如使用意大利体。举例如下:

\begin{quote}
    J'ai composé ce document en \emph{fourier} car j'en ai marre du \emph{times} trop vu, des \emph{garamond} un peu trop précieux et de l'\emph{helvetica} qui distingue mal les \emph{I} des \emph{l}. [les « I » des « l »]

    \begin{bil}
        我使用\emph{fourier}来撰写本文档,因为\emph{times}用得太多看得有点烦,\emph{garamond}看起来太贵气,\emph{helvetica}很难区分\emph{I}和\emph{l}(``I''和``l'')。
    \end{bil}
\end{quote}

\section{其他情况}

还有一些情况需要使用意大利体,但这与我们在Irisa不太能接触到它们:音符、戏剧中的舞台调度、议会会议记录中的标题等。

\section{意大利体与微版式}

有两个问题跟意大利体息息相关:

\subsection{意大利体中嵌套意大利体}

使用意大利体行文(如引用)时,如果想要进一步使用意大利体(如标示外来词),则回到使用罗马体上。举例如下:

\begin{quote}
    Pendibidu dit : « La clause de Horn (\emph{Horn-Fog}) est déclenchée dès que le niveau de récursivité dépasse 3 sur l'échelle de Richter. »

    Pendibidu dit : \emph{La clause de Horn \emph{(Horn-Fog)} est déclenchée dès que le niveau de récursivité dépasse 3 sur l'échelle de Richter.}

    \begin{bil}
        庞迪毕居说:\emph{一旦递归水平超过里氏标度3,就触发Horn子句\emph{(Horn-Fog)}。}
    \end{bil}
\end{quote}

注意,正因如此,在\LaTeX 中建议使用 \verb+\emph+ 而非 \verb+\textit+。

\subsection{意大利体、标点符号和括号}

传统上,意大利体后面的标点符号也要使用意大利体,但在引用语句时,这一点上可以保留一些模糊空间。举例如下:

\begin{quote}
    Un LDP, en anglais \emph{PDL,} est... < la virgule est en italique. >

    \begin{bil}
        LDP,英文写作\emph{PDL,}是……(后一个逗号使用了意大利体。)
    \end{bil}

    Qui a dit : \emph{Qui a cassé le vase de Soissons ?}\\
    Qui a dit : \emph{Clovis a cassé le vase de Soissons} ?

    \begin{bil}
        谁说:\emph{谁打碎了苏瓦松的花瓶?}\\
        谁说:\emph{克洛维斯打碎了苏瓦松的花瓶}?
    \end{bil}
\end{quote}

当括号中的内容全部为意大利体时,可以将括号设置为意大利,否则不使用。举例如下:

\begin{quote}
    les guillemets droits \emph{(double quote)} sont...< 2 parenthèses italiques>\\
    les guillemets droits (en anglais \emph{double quote}) sont...< 2 parenthèses droites >\\
    les guillemets droits (\emph{double quote} en anglais) sont...< 2 parenthèses droites >

    \begin{bil}
        直引号\emph{(double quote)}是……(两侧括号为意大利体)\\
        直引号(英文写作\emph{double quote})是……(两侧括号直立)\\
        直引号(\emph{double quote}英文这么写)是……(两侧括号直立)
    \end{bil}
\end{quote}

然而,意大利体或倾斜的符号有时会与罗马体重叠。例如,\emph{l}与右括号``)''可能重叠:``……\emph{l}\!)''。因此,需要(手动操作,但一些排版引擎可以自动完成)进行所谓的意大利体修正,也就是在两个字符间强制插入一个小的空白,来得到``……\emph{l}\,)''的效果。