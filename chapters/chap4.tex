\chapter{\textmd{\underline{加下线}、\textbf{加粗}还是\emph{使用意大利体}?}}

受制于技术原因,铅字时代的排印不存在添加下线的操作,它是到了打字机时代才出现的发明。实际上,由于无法打出粗体、意大利体,打字员大量使用下线。这种做法形成了一种文档风格,在今天还在影响着研究人员的报告,因为人们总是倾向于去重复曾经和当下看到的事物。然而,我们不能将下线和粗体、意大利体等同起来,从而互换使用。

\section{原则}

在分别介绍它们的用法之前,先来简要介绍一下阅读文档的模式。大体上来说,可以将文档分为两大类:线性阅读文档(如小说、文章、研究报告等),以及不必从头读到尾,而是去查询的文档(年鉴、目录、参考手册等)。前一种文档的排印应当是冷淡一些(在一页文本中,视线不会立刻被某个词或某个标记抢走);第二种文档的排印则应当跌宕起伏(使得视线可以快速锁定到正在寻找的内容)。当然,这种分类方式很不明确,一些作品可以同时属于两类,带有教学性质的作品尤为如此。

在详细说明前,先给出总体框架:

\begin{mdframed}
    粗体只用于两种情况:章节等的标题,以及在参考手册或目录中的入口标记。

    \noindent 意大利体表示区别:外语、强调、引用、作品名(包括参考资料中的书名和报刊名)。

    \noindent 通常情况下,没有任何理由使用下线。
\end{mdframed}

罗马体、粗体、意大利体、下线的侧重点不同:

\begin{itemize}
    \item 罗马体(字符笔直,如同此处的形式)是通常的行文字体,可读性一般较强。
    \item 意大利体是一种倾斜的字体(但其字母的形态通常与对应的罗马体不同,如“\emph{fa}”不是将“fa”倾斜而来)。这种倾斜的形态使得我们在罗马体的句子中看到意大利体时能够察觉区别(因此具有潜在的含义)。相反,当我们看到一页文字(比如此页)时,我们无法立刻注意到是否有使用意大利体的内容。因此,我们使用意大利体来体现不需要具有吸引视线功能的差异。\\
    一般认为,意大利体的可读性不如罗马体,整段的意大利体文字读起来比整段的罗马体文字吃力。此外,我们只在个别词上使用意大利体。如果使用得太多,意大利体将失去其“突出显示”的作用。
    \item 字符的粗体依靠其笔画和厚重程度实现。罗马体和意大利体多少都可以呈现出粗体的形态。对于罗马体的加粗变体通常称为“加重”(gras,直译为“油腻”)。将字体加粗,可以使其更显眼,从而更吸引目光。如果在一整段罗马体文字中夹带一个加粗的词(比如本段文字),在阅读时,你会立刻被\textbf{加粗}的词吸引。因此,我们使用加粗来吸引视线,尤其是指示文章结构(章节标题等),或者标识目录的入口、咨询手册中的定义等。\\
    加粗字符越多,则文本的可读性越差。因此,我们只在广告中使用很粗的文字。
    \item \underline{加下线}也可以吸引视线,且比加粗的效果更显著。此外,由于下线可能切断下探的笔画,字符会尤其难读。因此,极少有使用下线的充分理由。建议永远不要使用下线。特别要注意,我们已经被\emph{网}上超链接的下线养成了随手使用它的坏习惯。
\end{itemize}

现在,以颜色突出显示的成本很低(无论是在屏幕上还是在纸上)。因此,它也有被滥用的趋势,正如几年前切换字体的成本变低,导致我们曾滥用字体一样。我想说,请像谨慎使用粗体那样谨慎使用颜色,请避免使用蓝色文字,将蓝色留给网址。

\section{章节等标题}

字符的选择(字体、字重、字号)取决于版式布局,通常需要由专业人士来完成(参见\ref{??}%TODO
列出的几本书)。尽管如此,至少要遵循以下用法:

\begin{itemize}
    \item 通常,标题需要加粗。
    \item 层级更高的标题通常字号更大(例如,章标题字号为18,节标题为14,小节标题为12,正文为11,等等)。
    \item 这种渐进的选择应当足够显眼,使得即使没有序号,读者也能区分文档的结构,同时也不能过于夸张以至于扰乱阅读节奏(例如,请不要将节标题字号设置为18的同时将正文字号设置为10)。
    \item 除非情况特殊,否则标题层级不要超过三层或四层。此时,我们可以将最低级标题设置为意大利体,将正文设置为罗马体。
    \item 为加粗字符添加下线完全没必要(见图\ref{fig1}的第17行),没必要,没必要。
    \item 跟加粗同样重要的是标题前后的空白大小,但这是另一个问题了……
\end{itemize}

\section{入口和定义}

在目录或参考手册这样的文档中,推荐使用粗体来“牵制”视线。例如:

\begin{quote}
    并非所有参考资料的排版都相同。事实上,参考资料的排版取决于其性质:

    \textbf{图书:}书名需要使用意大利体,其余部分,包括(且必须包括)作者名、出版商、出版地点、出版日期,使用罗马体。
    
    \textbf{文章:}作者名使用罗马体;文章标题使用罗马体且置于引号间;期刊标题使用意大利体;其余部分,如卷号、序号、页码等,使用罗马体

    \textbf{论文:}与图书相同,但要将出版商替换为大学名。

    \textbf{等等。}
\end{quote}

定义可以使用意大利体或粗体。使用意大利体显得更审慎些,会更好地融入正文。使用粗体则更容易被识别(尤其是末尾用于跳转到定义所在页码的索引),但更吸引眼球也意味着会破坏线性阅读的体验。例如:

\begin{quote}
    Le système est organisé autour de couches : la première couche, autour du matériel, constitue ce qu’on appelle le \emph{noyau} du système. C’est lui qui réalise les \emph{échanges avec les périphériques} et qui gère les tâches (y compris la mémoire).

    Le système est organisé autour de couches : la première couche, autour du matériel, constitue ce qu’on appelle le \textbf{noyau} du système. C’est lui qui réalise les \textbf{échanges avec les périphériques} et qui gère les tâches (y compris la mémoire).

    \begin{bil}
        系统是围绕层组织的:围绕硬件的第一层构成了所谓的系统\emph{核心}。它执行\emph{与设备的交换}并管理任务(包括内存)。

        系统是围绕层组织的:围绕硬件的第一层构成了所谓的系统\textbf{核心}。它执行\textbf{与设备的交换}并管理任务(包括内存)。
    \end{bil}
\end{quote}

\section{计算机术语}

Algol 60的报告开创了在程序中加粗关键字的习惯,然而,使用意大利体本应是更好的做法。然而,习惯已经养成了,现在说这些太晚了!然而,我还是建议不要在指令、说明等文本中过度使用加粗字符。请比较以下两个表述:

\begin{quote}
    Cette fonction est plus difficile à mettre en œuvre. Nous créons deux types d’attributs \textbf{contact\_haut} et \textbf{contact\_bas} pour réaliser cette fonction \textbf{poser\_sur}. Comme nos primitives sont limitées au cube, à la sphère et au cylindre, \textbf{poser\_sur} a en fait trois possibilités...

    Cette fonction est plus difficile à mettre en œuvre. Nous créons deux types d’attributs \emph{contact\_haut} et \emph{contact\_bas} pour réaliser cette fonction \emph{poser\_sur}. Comme nos primitives sont limitées au cube, à la sphère et au cylindre, \emph{poser\_sur} a en fait trois possibilités...

    \begin{bil}
        此函数更难实现。我们将创建两个属性\textbf{contact\_haut}和\textbf{contact\_bas}来实现此\textbf{poser\_sur}函数。由于原函数限制在正方体、球体及圆柱上,\textbf{poser\_sur}实际由三种情况……

        此函数更难实现。我们将创建两个属性\emph{contact\_haut}和\emph{contact\_bas}来实现此\emph{poser\_sur}函数。由于原函数限制在正方体、球体及圆柱上,\emph{poser\_sur}实际上有三种情况……
    \end{bil}
\end{quote}

\section{术语}

\begin{itemize}
    \item 术语、我们想要突出显示的词、定义,总之,就是我们想要画重点(但不要真的画线上去)的内容,使用意大利体。\\
    \begin{quote}  
        Pour faire avancer la simulation, il faut que $\delta$ puisse déterminer que dans ces conditions \emph{aucun message ne sera émis} avant l’instant $\theta + 5$. Pour pouvoir libérer de la place en mémoire, on utilise la notion de \emph{temps virtuel global} : à un instant donné...
        \begin{bil}
            为了推进模拟,$\delta$必须能够确定在这些条件下,在时间$\theta + 5$之前\emph{不会发送任何消息}。为了释放内存空间,我们使用了\emph{全局虚拟时间}的概念:在给定时刻……
        \end{bil}
        
        Les électeurs sont donc invités à voter \emph{oui} lors du prochain scrutin.
        \begin{bil}
            因此,鼓励选民在下次投票中投\emph{赞成}票。
        \end{bil}
    \end{quote}
    \item 通常,用意大利体强调的内容也可以用引号强调,但使用引号时,意大利体应当改为罗马体。这是因为同时使用意大利体和引号相当于强调了两遍,是多余的。
\end{itemize}

\section{外来词}

\begin{itemize}
    \item 外来词应当使用意大利体(不要添加多余的引号)。例如:
    \begin{quote}
        Le projet a joué un rôle primordial dans la seconde \emph{International Conference on Supercomputing} qui s’est tenue à Saint-Malo en juin.
        \begin{bil}
            该项目在6月在圣马洛举行的第二届\emph{国际超级计算会议}上发挥了关键作用。
        \end{bil}

        Le système Mentoniezh (du breton \emph{ment}, mesure, et \emph{oniezh}, science de, c’est-à-dire géométrie)...
        \begin{bil}
            Mentoniezh系统(源自不列颠语\emph{ment} [意为测量]、\emph{oniezh},[意为“……的科学”],即地理)……
        \end{bil}

        Pour notre circuit, l’autocadencement (\emph{self-timing}) se révèle bien adapté...
        \begin{bil}
            对于我们的电路,自动计时(\emph{self-timing})非常适合……
        \end{bil}

        Jean Transen collabore avec le \emph{VLSI Research Group} de l’université d’Oxford.
        \begin{bil}
            让·特朗桑与牛津大学\emph{VLSI研究小组}合作。
        \end{bil}
    \end{quote}
    \item 外来的特定表达使用意大利体,如\emph{a capella}(纯人声演唱)、\emph{de facto}(事实上)、\emph{for ever}(永远)、\emph{honoris causa}(名誉上的)、\emph{ipso facto}(根据事实地)、\emph{manu militari}(武力地)、\emph{sine die}(不定期地)、\emph{up-to-date}(最新)等。
    \item 对应地,一些外来的表达在使用时保持罗马体。至于如何界定是罗马体和意大利体的界限,则有待讨论……例如:à priori
        \footnote{甚至还带重音符:参见http://langue-fr.net/spip.php?article128}
    (先验地)、ad hoc(专为此时的)、andante(行板)、curriculum vitæ(简历)、ex æquo(并列)、fair play(公平竞争)、mea culpa(捶胸悔过)、sketch(幕间短剧)、statu quo(现状)、vice versa(相反地)、etc.(等等)。(表示“等等”的etc.当然也是这个列表的一部分。)列表可能有不同,需要去看具体的版式要求。
    \item 在参考书目中的一些拉丁文(往往是首字母缩略词)需要使用意大利体(这一点在人文学科中比在计算机学科中更常见):\emph{passim}(及其他位置)、\emph{op. cit.}(同前)、\emph{infra}(下文)、\emph{ibidem}(同书),等等。
\end{itemize}

\section{引用}

