\chapter{标点符号}

作者在文章中输入标点符号的方式既是排印差错(如留白错误)的一大来源,也是编辑差错(如逗号的作用不明)的一大来源。

\section{标点符号和留白}

排印不是将符号放置在页面上的艺术,而是处理符号周边空白的艺术。精细的排印过程中,空白有很多种处理方式。在使用计算机排印的出版物中,可以近似地将空白分为两种,但也有必要明确区分这两种空白。

\subsection{空格}

我们考虑将空格分为两种(在强调一遍,排印术语espace为阴性名词;见%TODO
)

\begin{description}
    \item[可变空格](此处记为\verb*| |)这是单词间的常规空格。之所以称为可变,是因为系统可以轻微地改变它的宽度,以适应行长。行末的可变空格会被删除。\\
    在几乎所有系统中,按“空格键”都可以输入这种空格。
    \item[不可断空格](此处记为\verb|_|)这种空格有两个作用:
    \begin{itemize}
        \item 它的宽度通常比可变空格要小
            \footnote{这就要靠排印师的精细操作了,这里说的也是近似的情况(反例:冒号前的空格其实是是一个普通的不可断空格)。}
        ,
        \item 它永远不会被系统删除,系统也永远不会在它所在的地方断行(这也是其称为不可断的原因 )。\\
        这种空格要么通过特殊按键(如在Mac上按住\emph{Alt}和空格键)输入,要么通过代码输入(如\LaTeX 中使用波浪号)。一些排版引擎(如\LaTeX 、MS Word等)倾向于自动将必要的空格(如双标点前的空格)替换为不可断空格。
    \end{itemize}
\end{description}

\subsection{与标点搭配的是什么空格?}

表\ref{tab1} 给出了法文中这些空格的作用
    \footnote{盎撒人有着一套不同的系统:标点符号前不加空格,且句末的空格比法文中更宽。无论是完整的文章还是引用一句话,都应当尊重原文种的用法。例如,下面这条引用的写法是正确的(注意,冒号和问号前没有空格):\emph{Hypertext: where are the big systems?}}
。

需要注意以下几点:

\begin{table}
    \begin{center}
        \begin{tabular}{|l|l|}
            \hline
            句末 & mmm.\verb*| |mmm\\
            \hline
            缩写后 & mmm.\verb|_|mmm\\
            \hline
            逗号 & mmm,\verb*| |mmm\\
            \hline
            冒号 & mmm\verb|_|:\verb*| |mmm\\
            \hline
            分号 & mmm\verb|_|;\verb*| |mmm\\
            \hline
            感叹号 & mmm\verb|_|!\verb*| |mmm\\
            \hline
            问号 & mmm\verb|_|?\verb*| |mmm\\
            \hline
            连字符 & mmm-mmm\\
            \hline
            列表首的横线 & --\verb|_|mmm\\
            \hline
            左括号 & mmm\verb*| |(mmm\\
            右括号 & mmm)\verb*| |mmm\\
            \hline
            左引号 & mmm\verb*| |«\verb|_|mmm\\
            右引号 & mmm\verb|_|»\verb*| |mmm\\
            \hline
            插入语开始 & mmm\verb*| |---\verb|_|mmm\\
            插入语结束 & mmm\verb|_|---\verb*| |mmm\\
            \hline
            省略号 & mmm...\\
            & mmm,\verb|_|...\\
            \hline
        \end{tabular}
        \caption{插入标点符号。}
        \label{tab1}
        “m”表示任意字母,“\verb*| |”表示可变空格,“\verb|_|”表示不可断空格。
    \end{center}
\end{table}

\begin{table}
    \begin{center}
        \begin{tabular}{|l|||l|l|}
            \hline
            正确写法 & 错误写法 & 改正\\
            \hline
            ?, & ?. & ?\\
            !, & !. & !\\
            N., & N.. & N.\\
            etc., & etc.. & etc.\\
            & etc... & etc.\\
            & etc.... & etc. \\
            & ---. & .\\
            m, ... & m.... & m...\\
            m..., & m.... & m...\\
            \hline
        \end{tabular}
        \caption{对比逗号,使用句号时会出现收缩现象}
        \label{tab2}
    \end{center}
\end{table}

\begin{itemize}
    \item 句末点号标志着句子立刻结束,后面加可变空格。表\ref{tab2} 给出了句末点号前带有其他标点时的收缩现象。
    \item 表示缩写的点几乎总是紧随一个不可断空格,因为缩写的后面几乎总是紧跟不能分开的信息(见\ref{sec5.1.3} 小节)。而对于\textsc{Sncf}这样的首字母缩略词
    \footnote{当些那些不是首字母缩略词的词,比如ISO从来就不是\emph{International Standard Organization}的意思。}
    ,没有加点的写法有些无聊,但如果我们加上点,则只在最后加上一个(可变)空格,其他地方不加空格:Les T.G.V.\verb*| |de la S.N.C.F.\verb*| |sont...(S.N.C.F.\verb*| |的T.G.V.\verb*| |是……)
    \item 缩写点后面的标点符号通常保留(见表\ref{tab2} ),但涉及句点或省略号时,则需要删除前一个点。例如(注意两个F后面的缩写点不见了):
    \begin{quote}
        Citons la C.E.E., la C.I.A. et l’E.D.F... Sans oublier les S.D.F. Ainsi...

        \begin{bil}
            引用C.E.E.、C.I.A.和E.D.F.……但不要忘了S.D.F.。这样一来……
        \end{bil}
    \end{quote}
    \item \emph{et cætera}的缩写是“etc.”,不是“etc...”或“etc. etc.”,而这两种错误很常见。参见表\ref{tab2} 的收缩现象。此外,应当尽力避免etc.出现在行首。因此,推荐它的前面永远使用不可断空格:“\verb|_|etc.”。在\ref{sec5.2.4} 小节我们会看到,正常情况下,该空格前面还需要一个逗号。
    \item 省略号永远是三个点(实际上,应当使用一个独立的排印学符号“…”,由三个点组成,且中间带有间距,而并不是连续输入三个句点“...”);它会与句点或省略号混淆(见表\ref{tab2} )。
    \item 组成插入语的横杠如果处于句末,则会跟句点重复。因此,不应当写“ ... normalisation --- rappelons que \textsc{Sgml} est issu de \textsc{Gml} ---. La...”,应当这样写:
    \begin{quote}
        ... normalisation --- rappelons que \textsc{Sgml} est issu de \textsc{Gml}. La...
        
        \begin{bil}
            ……标准化——请记住,\textsc{Sgml} 源于\textsc{Gml}。那……
        \end{bil}
    \end{quote}
\end{itemize}

\subsection{不可断空格的其他作用}

出于阅读节奏的考量,文本中经常有两个元素不能分开的情况。为了避免在它们中间断行,可以使用不可断空格。

举例如下:

\begin{quote}
    \begin{tabular}{lll}
        \textbf{错误写法:} & ... le langage défini par N. & <N.\verb*+ +Bourbaki> \\
        & Bourbaki, ... & \\
        \textbf{正确写法:} & ... le langage défini par & <N.\verb|_|Bourbaki>\\
        & N. Bourbaki, ... & \\
    \end{tabular}

    \begin{bil}
        ……由N. Bourbaki定义的语言……
    \end{bil}
\end{quote}

\paragraph*{词间间隔} 处于可理解程度的充分考量,一些词或符号的中间不能断行,这种情况下,就会用到不可断空格。其主要情况如下(此处“\verb|_|”仍然代表不可断空格):

\begin{itemize}
    \item 在首字母缩略词和后面紧随的词之间,例如:M$^{\rm me}$\verb|_|Hugo、D.\verb|_|Knuth、le R.P.\verb|_|Durand;
    \item 在数字和其后用于确定其数量的词之间,例如:14\verb|_|francs(14法郎)、2\verb|_|€、1\verb|_|A、1\verb|_|ampère(1安培)、t.\verb|_|\textsc{vii}(卷\textsc{vii})、art.\verb|_|237(第237条)、fig.\verb|_|3(图3)、Louis\verb|_|\textsc{xiv}(路易十四)、pages\verb|_|23 à\verb|_|25(23~25~页)、23\verb|_|novembre\verb|_|1990(1990年11月23日)、98\verb|_|\%;
    \item 数字千分空:123\verb|_|456,789;
    \item 表示时间、经纬度、坐标,例如:À 9\verb|_|h\verb|_|14\verb|_|min\verb|_|23\verb|_|s, il naviguait par $10°$\verb|_|$20'$\verb|_|$30''$ de latitude\verb|_|N.(9时14分23秒,他在北纬$10°\ 20'\ 30''$航行。)关于\emph{度},参见注释
        \footnote{注意度的写法:角度$10°$、温度10\texttt{\_}℃。}
    ;
    \item 在表示序号的字数、数字及其后元素之间:Il y a deux cas\verb|_|: a)\verb|_|la fonction est récursive ou b)\verb|_|elle ne l’est pas. Nous allons voir 1$^{\rm o}$\verb|_|la fonction de Dirac et 2$^{\rm o}$\verb|_|le théorème de Schwartz.(有两种情况:a)\verb|_|该函数是循环的;b)\verb|_|该函数不是循环的。第一种称为迪拉克函数,第二种称为施瓦茨定理。)
\end{itemize}

\section{标点符号的作用}

正如前文所述,作者在文章中输入标点符号的方式是排印差错(如留白错误)的一大来源,但这种差错也会影响文本的理解,其中数逗号带来的问题最为典型。例如,请比较以下两个句子:

\begin{quote}
    Les étudiants de maîtrise qui ne suivront pas le cours du professeur Transen auront une séance supplémentaire d’analyse des données mardi prochain.
    \begin{bil}
        不参加特航桑教授课程的硕士生下周二将参加额外的数据分析会议。
    \end{bil}

    Les étudiants de maîtrise, qui ne suivront pas le cours du professeur Transen, auront une séance supplémentaire d’analyse des données mardi prochain.
    \begin{bil}
        硕士生啊,他们不参加特航桑教授的课程,他们下周二将参加额外的数据分析会议。
    \end{bil}
\end{quote}

对于第一种情况,仅有那些不去特航桑教授课程的人才会参加额外会议。对于第二种情况,一方面,没有学生会去参加特航桑教授的课程,另一方面,所有人都要参加额外的会议。

这里不想为大家上一整节语法课或语体课(尤其是讲逗号的作用),因此给出几条足以应对大多数情况的注意事项。参见\ref{ref15}%TODO
。

\subsection{句点}

这是表示句子结束的正常标识,没什么太多可说的,只需要强调:

\begin{itemize}
    \item 需要避免使用太长的句子;
    \item 从排印学层次来讲:
    \begin{itemize}
        \item 句点后需要带有一个普通空格(用英文打字时是两个);
        \item 标题末尾不加句点(标题不是句子)。
    \end{itemize}
\end{itemize}

\subsection{冒号}

冒号“:”用于引出紧随的句子。

\begin{itemize}
    \item 通常,冒号后紧随一个普通空格;但为了不将内容甩到行首,我们通常使用不可断空格:“mmm\verb|_|:\verb*+ +mmm”。
    \item 当引用内容由若干句子构成时,经常将引用内容置于引号间,且每个句子开头使用大写字母。否则,冒号后的内容一般不用大写。举例:
    
    \begin{quote}
        Il y a deux cas : le premier... ; le second..
        
        Pendibidu a conclu : « Bien que... normal. » Nous faisons notre...

        \begin{bil}
            由两种情况:第一……;第二……

            庞迪毕居总结道:“尽管……正常。”我们……
        \end{bil}
    \end{quote}
\end{itemize}

\subsection{分号}

尽管卡瓦纳写了很多反对使用分号的好文章,但法文中的分号比英文中的作用更丰富。分号连接的两个内容足够相关,以至于可以作为句子中的单一陈述,却也有足够的区别,足以称为两个不同的部分。

在排印学中,分号还有分隔列表中非整句的元素的作用(如上文\ref{sec5.2.1}小节中列表中第一个元素的末尾)。

注意:分号前需要插入狭窄的不可分空格。

\subsection{逗号}

对于一些科技文章作者来说,逗号带来了铺天盖地的麻烦。更准确地说,一些作者不正确地使用标点符号,给读者增添了铺天盖地的麻烦……然而,经过大幅简化,逗号的作用可以总结为两种情况。即使这种简化可能过于激进,我仍然相信逗号的这两个作用况足以覆盖绝大多数情况。

\paragraph{列表分隔符} 列表是一系列(大于一个)具有相同本质(形容词、表语、动词等)的元素。基本原则是将逗号用于相邻元素间,直到倒数第二个元素和最后一个元素之间,并在这里使用“et”(和)
    \footnote{相对地,英文只在仅含两个元素的列表中使用“\emph{and}”,对于超过两个元素的列表,则会在最后两个元素间使用“, \emph{and}”:$I_1$ \emph{and} $I_2$;$I_1, I_2, ... , I_{n-1}$, \emph{and} $I_n$。例如,在参考文献中:
    \begin{itemize}
        \item Thompson and Thomson, “A new case...”, ...
        \item Thompson, Thomson, and Thorson, “Another case...”, ...
    \end{itemize}
    }
。

\begin{mdframed}
    \subsubsection*{逗号:列表分隔符}

    \noindent 两个元素:<元素1> et <元素2>

    \noindent 三个元素:<元素1>, <元素2> et <元素3>

    \noindent $N$个元素:<元素1>, <元素2>, ... , <元素$N-1$> et <元素$N$>
\end{mdframed}

即使“etc.”是\emph{et cætera}(等等)的缩写,它也\emph{总是}带有逗号。因此,由“etc.”结尾的列表实际上应当由“, etc.”结尾(这里偏向使用“,\verb|_|etc.”,正如第\ref{p33}%TODO
所展示的)。

举例如下:

\begin{quote}
    Le projet a des relations suivies avec l’université de Fribourg et l’uni- versité de Neuchâtel. Des relations sont en cours d’établissement avec les universités de Lausanne, Bâle et Gruyère ainsi qu’avec les écoles polytechniques fédérales de Lausanne, Zürich, Lugano et Pully.

    Durant l’année 1989, on a étudié les méthodes de quantification vec- torielle adaptative et la compensation du mouvement sur un canal à bande réduite. En 1990, nous prévoyons de traiter l’approximation polygonale d’images contours, l’estimation des segments dynamiques, la reconstruction des segments en coordonnées plückeriennes et l’identification des polygones 3D par une méthode générale.

    Jean Transen a visité les universités de Caroline du Nord, du Sud, de l’Ouest, etc.

    \begin{bil}
        该项目与弗里堡大学和纳沙特尔大学保持着持续的关系。目前正在与洛桑、巴塞尔和格鲁埃本科以及洛桑、苏黎世、卢加诺和普利的联邦理工学院建立关系。

        1989年,研究了自适应矢量量化方法和窄带信道上的运动补偿。1990年,我们计划使用通用方法处理轮廓图像的多边形近似、动态段估计、普吕克坐标段重建和3D多边形识别。

        让·特航桑访问了北卡罗来纳州、南卡罗来纳州和西卡罗莱纳州等地的大学。
    \end{bil}
\end{quote}

注意:

\begin{enumerate}
    \item 在文学作品中,经常有将“et”替换为逗号的情况:L’attelage suait, soufflait, crachait.(马车出汗、喘气、嘎吱作响。)此外,当句子过长时,有时可以使用“, et”,尤其是连词“et”已经在列表元素中出现时:Nous nous intéressons à deux points essentiels : la décomposition pyramidale qui est évaluée en termes d’entropie, de complexité et de qualité de reconstruction, et l’emploi d’algorithmes multigrilles.(我们对两个关键点感兴趣:根据熵、复杂性和重建质量评估的锥体分解,以及使用多网格算法。)
    \item “et”的使用方法同样适用于“ou”(或),通常也适用于“ni”(也非)。
\end{enumerate}

\paragraph{充当括号}

逗号也可以充当括号来使用。因此,如下两个句子是等价的:

\begin{quote}
    这些信息(即标签+标识符)足以……

    这些信息,即标签+标识符,足以……
\end{quote}

这里不要推广得太远,以下两个句子也是等价的:

\begin{quote}
    Nous démarrons, par ailleurs, une réflexion sur l’intégration de toutes les techniques qui, d’une part, ...

    \begin{bil}
        此外,一方面,我们启动了对整合所有技术的思考……
    \end{bil}

    Nous démarrons (par ailleurs) une réflexion sur l’intégration de toutes les techniques qui (d’une part) ...

    \begin{bil}
        我们(此外)(一方面)启动了对整合所有技术的思考……
    \end{bil}
\end{quote}

\begin{mdframed}
    \subsubsection*{逗号充当括号}

    \begin{tabular}{rcl}
        Xxxx (mmmmmmm) yyy. & = & Xxx, mmmmmmm, yyy.\\
        mmm. (Mmm) xxxxx. & = & mmm. Mmm, xxxxx.\\
        Xxxx (mmmmmm). Mmm & = & Xxxx, mmmmmm. Mmm
    \end{tabular}
\end{mdframed}

这里说的充当括号,是指:

\begin{itemize}
    \item 用于从句,例如:
    \begin{quote}
        若将相关成分记为$C$,由欧拉公式得……

        我们将每个处理器,在保证功能同步的情况下,都连接到一个缓冲器……
    \end{quote}
    \item 用于插入语,例如:
    \begin{quote}
        Ada语言,一种古早的编程语言,是……

        Oda,全称为\emph{Office Document Architecture},是一个范式……
    \end{quote}
    \item 作为空间、时间上或出于平衡考虑的一种指示。此外,一些连词也可以置于两个逗号之间。例如:
    \begin{quote}
        对于身份识别的情况,我们尝试证明……

        On a, d’une part, une union finie de courbes de Jourdan et, d’autre part, une union finie de régions simplement connexes.
        \begin{bil}
            我们一方面有一个若尔当曲线的有限并集,另一方面有一个简单相连区域的有限并集。
        \end{bil}

        因此,可以说……

        当然,我们没有理由……

        On ne connaît pas, aujourd’hui, de solution...
        \begin{bil}
            今天,我们还不知道解决方案……
        \end{bil}
    \end{quote}
\end{itemize}

\subsubsection{注意}

\begin{itemize}
    \item 我们不能给出诸如“\emph{et}前不能加逗号”这样的规则,因为以上两种作用可能会同时存在,尤其是并列连词本身也具有强调意义(并且$=$此外):“我们可以这样做,并且我们已经这样做了,……”
    \item 关系从句不加逗号(除非作为插入语):
    \begin{quote}
        Les forces qui engendrent cet effet sont de nature...
        Les forces, qui engendrent cet effet, sont de nature...
        \begin{bil}
            产生这种效果的力本质上是……
        \end{bil}
    \end{quote}
    \item 如果删除充当括号的逗号和用于列表的逗号,那么主语和谓语之间不应当还有任何逗号残留。
\end{itemize}

\subsection{使用句号、引号或意大利体}

插入语中间或后边用什么标点符号?原则是,如果标点符号属于插入语(比如引用句子),那么处理它的方式和插入语的其他部分相同:放入引号间或使用意大利体。此外,应用相连标点符号的收缩规则(见表\ref{tab1}%TODO
)时不考虑这些标记(如意大利体的末尾或结束引号)。有以下几个示例:

\begin{description}
    \item[错误:] Pendibidu demanda : \emph{La clause de Horn est-elle toujours valide ?}. Il répondit \emph{Non !}. Mais...
    \item[正确:] Pendibidu demanda : \emph{La clause de Horn est-elle toujours valide ?} Il répondit \emph{Non !} Mais...
    \item[错误:] Pendibidu demanda : « La clause de Horn est-elle toujours valide~?~». Il répondit « Non ! ». Mais...
    \item[正确:] Pendibidu demanda : « La clause de Horn est-elle toujours valide~?~» Il répondit « Non ! » Mais...
\end{description}

\begin{bil}
    \begin{description}
        \item[错误:]  庞迪毕居问道:\emph{Horn子句仍然有效吗?}。他回答\emph{不是!}。但……
        \item[正确:] 庞迪毕居问道:\emph{Horn子句仍然有效吗?}他回答\emph{不是!}但……
        \item[错误:] 庞迪毕居问道:“Horn子句仍然有效吗?”。他回答“不是!”。但……
        \item[正确:] 庞迪毕居问道:“Horn子句仍然有效吗?”他回答“不是!”但……
    \end{description}
\end{bil}

\subsection{不要滥用}

最后一条注意事项:标点符号有点像编程中括号的作用(尤其是对于Lisp语言来说),但人类没法适应过多层级的递归。请使用短句,不要在同一句话里表达过多内容。请想起你的读者,不论用什么方式,阅读你的文字都是一件苦差事(\emph{语自}高德纳[Donald Knuth])。