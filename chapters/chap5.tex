\chapter{标点符号}

作者在文章中输入标点符号的方式既是排印差错(如留白错误)的一大来源,也是编辑差错(如逗号的作用不明)的一大来源。

\section{标点符号和留白}

排印不是将符号放置在页面上的艺术,而是处理符号周边空白的艺术。精细的排印过程中,空白有很多种处理方式。在使用计算机排印的出版物中,可以近似地将空白分为两种,但也有必要明确区分这两种空白。

\subsection{空格}

我们考虑将空格分为两种(在强调一遍,排印术语espace为阴性名词;见%TODO
)

\begin{description}
    \item[可变空格](此处记为\verb*| |)这是单词间的常规空格。之所以称为可变,是因为系统可以轻微地改变它的宽度,以适应行长。行末的可变空格会被删除。\\
    在几乎所有系统中,按“空格键”都可以输入这种空格。
    \item[不可断空格](此处记为\verb|_|)这种空格有两个作用:
    \begin{itemize}
        \item 它的宽度通常比可变空格要小
            \footnote{这就要靠排印师的精细操作了,这里说的也是近似的情况(反例:冒号前的空格其实是是一个普通的不可断空格)。}
        ,
        \item 它永远不会被系统删除,系统也永远不会在它所在的地方断行(这也是其称为不可断的原因 )。\\
        这种空格要么通过特殊按键(如在Mac上按住\emph{Alt}和空格键)输入,要么通过代码输入(如\LaTeX 中使用波浪号)。一些排版引擎(如\LaTeX 、MS Word等)倾向于自动将必要的空格(如双标点前的空格)替换为不可断空格。
    \end{itemize}
\end{description}

\subsection{与标点搭配的是什么空格?}

表\ref{tab1} 给出了法文中这些空格的作用
    \footnote{盎撒人有着一套不同的系统:标点符号前不加空格,且句末的空格比法文中更宽。无论是完整的文章还是引用一句话,都应当尊重原文种的用法。例如,下面这条引用的写法是正确的(注意,冒号和问号前没有空格):\emph{Hypertext: where are the big systems?}}
。

需要注意以下几点:

\begin{table}
    \begin{center}
        \begin{tabular}{|l|l|}
            \hline
            句末 & mmm.\verb*| |mmm\\
            \hline
            缩写后 & mmm.\verb|_|mmm\\
            \hline
            逗号 & mmm,\verb*| |mmm\\
            \hline
            冒号 & mmm\verb|_|:\verb*| |mmm\\
            \hline
            分号 & mmm\verb|_|;\verb*| |mmm\\
            \hline
            感叹号 & mmm\verb|_|!\verb*| |mmm\\
            \hline
            问号 & mmm\verb|_|?\verb*| |mmm\\
            \hline
            连字符 & mmm-mmm\\
            \hline
            列表首的横线 & --\verb|_|mmm\\
            \hline
            左括号 & mmm\verb*| |(mmm\\
            右括号 & mmm)\verb*| |mmm\\
            \hline
            左引号 & mmm\verb*| |«\verb|_|mmm\\
            右引号 & mmm\verb|_|»\verb*| |mmm\\
            \hline
            插入语开始 & mmm\verb*| |---\verb|_|mmm\\
            插入语结束 & mmm\verb|_|---\verb*| |mmm\\
            \hline
            省略号 & mmm...\\
            & mmm,\verb|_|...\\
            \hline
        \end{tabular}
        \caption{插入标点符号。}
        \label{tab1}
        “m”表示任意字母,“\verb*| |”表示可变空格,“\verb|_|”表示不可断空格。
    \end{center}
\end{table}

\begin{table}
    \begin{center}
        \begin{tabular}{|l|||l|l|}
            \hline
            正确写法 & 错误写法 & 改正\\
            \hline
            ?, & ?. & ?\\
            !, & !. & !\\
            N., & N.. & N.\\
            etc., & etc.. & etc.\\
            & etc... & etc.\\
            & etc.... & etc. \\
            & ---. & .\\
            m, ... & m.... & m...\\
            m..., & m.... & m...\\
            \hline
        \end{tabular}
        \caption{对比逗号,使用句号时会出现收缩现象}
        \label{tab2}
    \end{center}
\end{table}

\begin{itemize}
    \item 句末点号标志着句子立刻结束,后面加可变空格。表\ref{tab2} 给出了句末点号前带有其他标点时的收缩现象。
    \item 表示缩写的点几乎总是紧随一个不可断空格,因为缩写的后面几乎总是紧跟不能分开的信息(见\ref{sec5.1.3} 小节)。而对于\textsc{Sncf}这样的首字母缩略词
    \footnote{当些那些不是首字母缩略词的词,比如ISO从来就不是\emph{International Standard Organization}的意思。}
    ,没有加点的写法有些无聊,但如果我们加上点,则只在最后加上一个(可变)空格,其他地方不加空格:Les T.G.V.\verb*| |de la S.N.C.F.\verb*| |sont...(S.N.C.F.\verb*| |的T.G.V.\verb*| |是……)
    \item 缩写点后面的标点符号通常保留(见表\ref{tab2} ),但涉及句点或省略号时,则需要删除前一个点。例如(注意两个F后面的缩写点不见了):
    \begin{quote}
        Citons la C.E.E., la C.I.A. et l’E.D.F... Sans oublier les S.D.F. Ainsi...

        \begin{bil}
            引用C.E.E.、C.I.A.和E.D.F.……但不要忘了S.D.F.。这样一来……
        \end{bil}
    \end{quote}
    \item \emph{et cætera}的缩写是“etc.”,不是“etc...”或“etc. etc.”,而这两种错误很常见。参见表\ref{tab2} 的收缩现象。此外,应当尽力避免etc.出现在行首。因此,推荐它的前面永远使用不可断空格:“\verb|_|etc.”。在\ref{sec5.2.4} 小节我们会看到,正常情况下,该空格前面还需要一个逗号。
    \item 省略号永远是三个点(实际上,应当使用一个独立的排印学符号“…”,由三个点组成,且中间带有间距,而并不是连续输入三个句点“...”);它会与句点或省略号混淆(见表\ref{tab2} )。
    \item 组成插入语的横杠如果处于句末,则会跟句点重复。因此,不应当写“ ... normalisation --- rappelons que \textsc{Sgml} est issu de \textsc{Gml} ---. La...”,应当这样写:
    \begin{quote}
        ... normalisation --- rappelons que \textsc{Sgml} est issu de \textsc{Gml}. La...
        
        \begin{bil}
            ……标准化——请记住,\textsc{Sgml} 源于\textsc{Gml}。那……
        \end{bil}
    \end{quote}
\end{itemize}

\subsection{不可断空格的其他作用}