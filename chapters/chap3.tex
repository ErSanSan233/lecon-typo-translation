\chapter{大写字母的作用}

\section{大写字母的滥用}

在阅读其他人写的科技文档时,最让我恼火的莫过于滥用大写字母。以下是一个典型的句子,出自我们的活动报告:

\begin{quote}
    Jean TRANSEN, Maître de Conférences en Analyse des Données à l'Université de Nancy (Bien connue de la Communauté Scientifique Internationale) a donné, lors du Séminaire de Biologie Informatique du Mardi 23 Juin, une conférence sur les Applications de l'Intelligence Artificielle à l'emploi de la Télévision Haute Définition en Robotique Avancée.

    \begin{bil}
        南希大学数据分析高级讲师让·特拉桑(国际科学界知名)在6月23日(星期二)的计算机生物学研讨会上就人工智能在高级机器人技术的高清电视中的应用发表了演讲。
    \end{bil}
\end{quote}
    
这句话中用了31个大写字母,而如果大小写规则应用正确,则只需要3个(只有Jean、Transen和Nancy需要大写)。没错,没错……我们将在\ref{sec3.6}节重新讨论这件事。

大写字母
    \footnote{在排印学中,我们将大写字母称为“capitale”,将小字写字母称为“bas de casse”。参见稍后的注释\ref{note15}。}
的使用充满了特殊情况,但我们可以依照如下的方法来为基本规则分类。在开始之前,再次强调,大写字母和小型大写字母都应该正确地带有变音符号。

\section{大写字母需要带变音符号吗?}

尽管我们在小学是可能(错误地)学过,大写字母不加变音符号,尽管一些编辑或期刊工作人员不会系统地处理变音符号
    \footnote{但会保持一致性。具体的原因(参见参考资料[3, 6]%TODO
    )跟语言学大写(majuscule)和排印学大写(capitale)有关。例如,VICTOR HUGO包含10个排印学大写字母,但只有2个是语言学的大写字母。}
,但下面这句话需要作为一条规则来遵守:在今天,没有任何理由来不为大写字母添加变音符号。相反,我们有太多地理由来为系统地添加变音符号。在这里给出了3个例子来证明带有变音符号的大写字母有助于文字的理解,实际上,还有更多例子能佐证这一点
    \footnote{有很多实用的例子(参见http://www. synec-doc.be/doc/accents2.htm)可以证明,使用小写字母(因为一些短语通常没有必要使用大写字母)或结合背景知识可以减少歧义。例如,SABLE SALE写在海滩入口的牌子上代表“沙子很脏”(sable sale),而写在饼干包装上代表“咸酥口味”(sablé salé)。相反,戏剧海报上写着CLAUDE S'EST TUE并不会减少歧义,可能是“克劳德自杀了”(Claude s'est tué),也可能是“克劳德沉默了”(Claude s'est tue)……}
:

\begin{enumerate}
    \item 在Minitel上发出以下消息的人收到了很多年长男性的打扰,因为很多人将“要求年龄相仿”(même âge)读成了“大龄人士也可”(même âgé):
    \begin{quote}
        CHOUETTE NANA, 18 ANS, CHERCHE MEC, MEME AGE, ... 
        \begin{bil}
            舒埃特·娜娜,18岁,寻找男士,要求年龄相仿……
        \end{bil}
    \end{quote}
    \item 在瑞士洛桑,有一条很陡的街道叫做le Petit-Chêne,在街道的入口有个牌子写着DANGER MARCHE(danger marche,行路危险/danger marché,危险集市)。看起来它只在周三起作用,因为周三才有集市。
    \item 本文档封面上的图片来自蒙帕纳斯的一家海鲜餐厅的招牌,上面的“海鳗”(le congre)是否写多了一个\emph{S}%TODO 封面
        \footnote{答案是没有。实际上,它写的是“议会”(le Congrès),只不过\emph{CONGRÈS}的\emph{È}缺了重音符。这里的背景知识没有给出:这家餐厅只是一家分店,而它的总店位于巴黎的Porte Maillot大道,靠近会议中心(Palais des congrès)。}
    ?
\end{enumerate}

如果你论文的指导老师认为,大写字母加不加变音符号只是个人喜好,请向他出示以下内容:

\begin{quote}
    « On veillera à utiliser systématiquement les capitales accentuées, y compris la préposition À. »\\
    \emph{Lexique des règles typographiques en usage à l'Imprimerie nationale}, Paris, 2004, p. 12.
    \begin{bil}
        “注意,应始终使用带有变音符号的大写字母,包括介词À.”\\
        ——《国家印刷馆实用排版规则汇编》,巴黎,2004,第12页。
    \end{bil}
\end{quote}

\section{大写字母的几个作用}

\subsection{句子和括号}

这里有几个原则:

\begin{itemize}
    \item 句子的第一个字母应用大写,
    \item 插入语,尤其是以括号形式插入的内容不是句子,因此不以大写字母开头,
    \item 一般地,冒号后为小写字母。
\end{itemize}

举例如下:

\begin{description}
    \item[错误写法]  Ces transformations dépendent du type de nœud (Qu’il soit terminal ou pas) : Tant que...
    \item[正确写法] Ces transformations dépendent du type de nœud (qu’il soit terminal ou pas) : tant que...
\end{description}

\begin{bil}
    这些转换取决于节点的类型(无论其是否终端):只要……
\end{bil}

注意,美国人喜欢将注释整句放入括号,并且将其视为一句话。在法文中,没有理由这样做。举例如下:

\begin{description}
    \item[错误写法] ... et le temps d’exécution est négligeable. (On ne tient pas compte du cas où $v = 0$.) Si...
    \item[正确写法]  ... et le temps d’exécution est négligeable (on ne tient pas compte du cas où $v = 0$). Si...
    \item[更佳写法] ... et le temps d’exécution est négligeable ; on ne tient toutefois pas compte du cas où $v = 0$. Si...
\end{description}

\begin{bil}
    ……并且执行时间可忽略不计(且我们不考虑$v = 0$的情况)。如果……
\end{bil}

\subsection{文章及章节标题等}

\begin{itemize}
    \item 仅有标题的第一个字母需要大写。
\end{itemize}

举例如下(同时请参见图\ref{fig1}第1行):

\begin{description}
    \item[错误写法] Transparence de la Transmission de Message Asynchrone
    \item[正确写法] Transparence de la transmission de message asynchrone
\end{description}

\begin{bil}
    异步信息传输的透明性
\end{bil}

注意,前一个标题写得“美里美气”。

\subsection{列表}

大体上,我们可以将列表分为两类,一类一个独立句子的一部分,另一类本身就由多个句子组成:

\begin{enumerate}
    \item 在句子中的列表的元素以小写字母开始,以逗号(或分号)结束,除了最后一个元素要实用句点结束,因为同时作为句子的结尾;
    \item 由多个句子组成的列表的元素需要遵循单独段落的规则,即以大写字母开始,以句点结束。
\end{enumerate}

上面这个列表本身满足情况1(同时要注意,紧接数字的句点之后不需要使用大写字母)。除此之外,以下是两个其他例子:

\begin{quote}
    Les méthodes de déverminage sont basées sur
    \begin{itemize}
        \item la récolte d’événements~;
        \item la sauvegarde de l’état du programme à intervalles réguliers~;
        \item l’intégration des tâches au programme~;
    \end{itemize}
    mais on aurait une autre classification en se plaçant du point de vue utilisateur.
    \begin{bil}
        清理方法基于:
        \begin{itemize}
            \item 事件的收集;
            \item 程序状态的定期保存;
            \item 程序中任务的纳入;
        \end{itemize}
        但换到用户的视角,则有另一种分类方法。
    \end{bil}
    
\end{quote}

\begin{quote}
    Deux types d’événements sont à considérer.
    \begin{itemize}
        \item Les événements prédéfinis. La trace en est générée par le noyau.
        \item Les événements utilisateur. L’heure, par exemple, sera associée à l’événement en question.
    \end{itemize}
    Une fois stockés sur fichier, ces événements...
    \begin{bil}
        需要考虑两种事件:
        \begin{itemize}
            \item 预定义事件。其轨迹由内核生成。
            \item 用户事件。例如,时间需要与所涉及的事件相关。
        \end{itemize}
        一旦存储为文件,这些事件……
    \end{bil}
\end{quote}

\subsection{首字母缩略词}

这里说的更多是目前的趋势,而不是死规则:

\begin{itemize}
    \item 首字母缩略词中间不再总要加句点,
    \item 需要尊重首字母缩略词拥有方的使用习惯(尤其是否在大写字母上加变音符号),
    \item 在首字母缩略词可以整体发音时,仅将第一个字母大写,
    \item 如果首字母缩略词需要我们逐字母读出,则更偏向于使用小型大写字母,
    \item 一些首字母缩略词有时会构成徽标(比如法国电力集团[Électricité de France]的徽标是EDF而非\textsc{Édf}),我们遵从徽标上的写法,一些品牌名也如此(如iPod)。
\end{itemize}

以下是一些例子:Irisa、Ifsic、\textsc{Sncf}、EDF、\textsc{Cee}、Afcet、Greco、Sorep,等等。

\section{大写字母的语义学作用}

目前为止,规则都只受制于单词的位置和一些缩写的约定。滥用大写字母的一个原因是认为大写字母可以由强调或区分语义的作用。实际上,强调和区分语义应当交由其他排印学方式来完成(如使用意大利体)。

基本的规则如下。首字母大写可以表示

\begin{itemize}
    \item 专有名词,
    \item 跟专有名词等价的名称,
    \item 一定程度的尊重(用于客套)。
\end{itemize}

\subsection{专有名词}

\begin{itemize}
    \item 首字母大写。
\end{itemize}

但是,只有首字母应当大写。因此,应当写J.-M. Pendibidu或Jean-Marie Pendibidu,而不要写J.M. PENDIBIDU(见图\ref{fig1}第4行)。在文章标题中或作为参考书目时,我们倾向于使用小型大写字母来表示作者的姓,如Donald \textsc{Knuth}。

类似地,假名(如Raphaël le Tatoué[纹身人拉斐尔])、地名(如la Picardie septentrionale[北皮卡第大区])、非通用的商品名(如Bull vend des klaxons à Renault pour ses jeeps[公牛向雷诺出售吉普车喇叭])等也可以大写。

可以参与构成名称的冠词也使用大写,如les œuvres de La Fontaine(拉封丹的作品)。

\subsection{跟专有名词等价的名称}

这里需要区分3种情况:

\begin{enumerate}
    \item 标明独特性的普通名词(如la Bibliothèque nationale[国家图书馆]是唯一的;类似地,la Culture[文化部]表示一个部门而非宽泛的文化概念)可以看作专有名词,使用大写。当形容词对名称起到限定作用时,在名词前的形容词使用大写(如le Nouveau Monde[新世界]),在名词后的形容词使用小写(如l'Empire romain[罗马帝国])。
    \item 如果独特性是由专有名词表达的,则只有专有名词使用大写,其他单词使用小写:la bibliothèque d'Alexandrie(亚历山大图书馆)、la bibliothèque Mazarine(马萨林图书馆)。
    \item 一些普通名词可以作为专有名词使用,尤其是交通工具或作品名(这两种名词需要使用意大利体),如\emph{La Belle Poule}(贝勒·普尔[本义为美丽母鸡]号护卫舰)、\emph{Les mains sales}(戏剧《脏手》;对于这种情况,只有第一个冠词的首字母大写),以及节日名(如le Mardi gras[狂欢节,本义为油腻星期二])、政党或其他组织名(如le Parti communiste français[法国共产党];需要符合其措辞要求)等。
\end{enumerate}

所有的排印手册对会给出一系列不同情况的列表,但无疑更重要的是要注意\emph{不需要大写的情况}!参见\ref{sec3.5}节。

\subsection{出于礼貌或尊敬的大写}

\begin{itemize}
    \item 出于礼节,需要使用大写。
\end{itemize}

示例如下:Croyez, Cher Monsieur, en l’expression...(请相信,亲爱的先生,有句话……)

\section{延伸}

对于前文没有提到的情况,不使用大写字母。尤其是以下情况:

\begin{itemize}
    \item 不唯一的组织,无论是不是国有的,都不大写:l’université de Rennes(雷恩大学)、l’académie de Poitiers(普瓦捷学院)、 le conseil municipal de Rezé(雷泽市议会)、l’observatoire de Meudon(默东天文台)、la mairie de Paris(巴黎市政厅)等;有时我们将其视作法律实体或机构,此时则倾向于使用大写:... l’Université de Rennes-2 a décidé...(……雷恩二大决定……);
    \item 星期和月份不大写:mardi 25 janvier(1月25日星期二);
    \item 头衔或身份使用小写字母: le président de la République(法国总统)、le pape(神父)、l’ayatollah Khomeini(阿亚图拉霍梅尼)、le professeur Dupond(迪蓬教授)、le recteur Durand(迪朗院长)、le général Avendre(阿文德将军)、le ministre de l’Éducation nationale(教育部长)等
        \footnote{请看看《世界报》是怎么报世界的。跟人们普遍的看法相反,通常来说,报刊可以很好地应用法文排版,即使是地区报刊。}
    。
\end{itemize}

\section{回到第一个例子}
