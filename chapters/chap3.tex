\chapter{大写字母的作用}

\section{大写字母的滥用}

在阅读其他人写的科技文档时,最让我恼火的莫过于滥用大写字母。以下是一个典型的句子,出自我们的活动报告:

\begin{quote}
    Jean TRANSEN, Maître de Conférences en Analyse des Données à l'Université de Nancy (Bien connue de la Communauté Scientifique Internationale) a donné, lors du Séminaire de Biologie Informatique du Mardi 23 Juin, une conférence sur les Applications de l'Intelligence Artificielle à l'emploi de la Télévision Haute Définition en Robotique Avancée.

    \begin{mdframed}
        \footnotesize
        南希大学数据分析高级讲师让·特拉桑(国际科学界知名)在6月23日(星期二)的计算机生物学研讨会上就人工智能在高级机器人技术的高清电视中的应用发表了演讲。
    \end{mdframed}
\end{quote}
    
这句话中用了31个大写字母,而如果大小写规则应用正确,则只需要3个(只有Jean、Transen和Nancy需要大写)。没错,没错……我们将在\ref{sec3.6}节重新讨论这件事。

大写字母
    \footnote{在排印学中,我们将大写字母称为“capitale”,将小字写字母称为“bas de casse”。参见稍后的注释\ref{note15}。}
的使用充满了特殊情况,但我们可以依照如下的方法来为基本规则分类。在开始之前,再次强调,大写字母和小型大写字母都应该正确地带有变音符号。

\section{大写字母需要带变音符号吗?}

尽管我们在小学是可能(错误地)学过,大写字母不加变音符号,尽管一些编辑或期刊工作人员不会系统地处理变音符号
    \footnote{但它们会保持一致性。具体的原因(参见参考资料[3, 6]%TODO
    )跟语言学大写(majuscule)和排印学大写(capitale)有关。例如,VICTOR HUGO包含10个排印学大写字母,但只有2个是语言学的大写字母。}
,但下面这句话需要作为一条规则来遵守:在今天,没有任何理由来不为大写字母添加变音符号。相反,我们有太多地理由来为系统地添加变音符号。在这里给出了3个例子来证明带有变音符号的大写字母有助于文字的理解,实际上,还有更多例子能佐证这一点
    \footnote{有很多实用的例子(参见http://www. synec-doc.be/doc/accents2.htm)可以证明,使用小写字母(因为一些短语通常没有必要使用大写字母)或结合背景知识可以减少歧义。例如,SABLE SALE写在海滩入口的牌子上代表“沙子很脏”(sable sale),而写在饼干包装上代表“咸酥口味”(sablé salé)。相反,戏剧海报上写着CLAUDE S'EST TUE并不会减少歧义,可能是“克劳德自杀了”(Claude s'est tué),也可能是“克劳德沉默了”(Claude s'est tue)……}
:

\begin{enumerate}
    \item 在Minitel上发出以下消息的人收到了很多年长男性的打扰,因为很多人将“要求年龄相仿”(même âge)读成了“大龄人士也可”(même âgé):
    \begin{quote}
        CHOUETTE NANA, 18 ANS, CHERCHE MEC, MEME AGE, ... 
        \begin{mdframed}
            \footnotesize
            舒埃特·娜娜,18岁,寻找男士,要求年龄相仿……
        \end{mdframed}
    \end{quote}
    \item 在瑞士洛桑,有一条很陡的街道叫做le Petit-Chêne,在街道的入口有个牌子写着DANGER MARCHE(danger marche,行路危险/danger marché,危险集市)。看起来它只在周三起作用,因为周三才有集市。
    \item 本文档封面上的图片来自蒙帕纳斯的一家海鲜餐厅的招牌,上面的“海鳗”(le congre)是否写多了一个\emph{S}%TODO 封面
        \footnote{答案是没有。实际上,它写的是“议会”(le Congrès),只不过\emph{CONGRÈS}的\emph{È}缺了重音符。这里的背景知识没有给出:这家餐厅只是一家分店,而它的总店位于巴黎的Porte Maillot大道,靠近会议中心(Palais des congrès)。}
    ?
\end{enumerate}

如果你论文的指导老师认为,大写字母加不加变音符号只是个人喜好,请向他出示以下内容:

\begin{quote}
    « On veillera à utiliser systématiquement les capitales accentuées, y compris la préposition À. »\\
    \emph{Lexique des règles typographiques en usage à l'Imprimerie nationale}, Paris, 2004, p. 12.
    \begin{mdframed}
        \footnotesize
        “注意,应始终使用带有变音符号的大写字母,包括介词À.”\\
        《国家印刷馆实用排版规则汇编》,巴黎,2004,第12页。
    \end{mdframed}
\end{quote}

\section{大写字母的几个作用}

\subsection{句子和括号}

这里有几个原则:

\begin{itemize}
    \item 句子的第一个字母应用大写,
    \item 
\end{itemize}